\section{Validation}
\label{sec:validation}

In this section we provide an overview of the messages, and programs used for the validation of the prototype. The transmission of the following message samples requires the system to be started, and the operations until the deployment of the tenant operator's \ac{SA} to be executed. These messages are provided in a soapUI test suite shipped with the prototype. From the moment when the tenant deploys the \ac{SA}, the configuration of the frontend and backend data stores can be associated to it. The data store configuration is done through the execution of operations which are accessible through a Web service interface. Therefore, either the \term{Cloud Data Migration Tool}, or the tenant can configure their database connections through the \ac{ESB} to the backend Cloud data store where his data is migrated to.
 
\lstinputlisting[label={lst:testaddds},caption={[Add Source and Target Data Source Sample Request]Add Source and Target \ac{SQL} Data Source \ac{SOAP} over \ac{HTTP} sample request.},style=xml]{./gfx/addds.xml}

The message described in Listing \ref{lst:testaddds} contains the needed information for registering and providing access to the tenant operator A in the MySQL CDASMix Proxy, and connecting with an \ac{SQL} database system in a backend Cloud data store. The tenant operator must authenticate with JBIMulti2 in order to register the connection meta-data, e.g. protocol, URL, database type, access credentials, etc. In Listing \ref{lst:testaddds} the tenant operator A, which is a user of the tenant Taxi Company, registers for the endpoint configuration described in the \term{servicemix.cdasmix.jbi.camelSA} a connection with a MySQL 5.1.3 database system instance in the Amazon RDS infrastructure. When more than one target database system wants to be attached to an existing source data source registered in the system, this provides the \term{AttachTargetDataSource} operation, which has as pre-condition the existence of the specified source data source.

\lstinputlisting[label={lst:testaddmis},caption={[Add Source and Target Main Information Structure Sample Request]Add Source and Target  \ac{SQL} Main Information Structure \ac{SOAP} over \ac{HTTP} sample request.},style=xml]{./gfx/addmaininfosql.xml}

When the source and target data source configuration are registered, the tenant operator (or through the \term{Cloud Data Migration Tool}) can register the storage structure meta-data of his database. For example, for a MySQL database system instance in Amazon RDS, the tenant operator must provide the tables' name (see Listing \ref{lst:testaddmis}), or for a bucket stored in a backend Google Cloud Storage container the tenant must provide the bucket identifier.

\lstinputlisting[label={lst:testaddsis},caption={[Add Source and Target Secondary Information Structure Sample Request]Add Source and Target \ac{NoSQL} Secondary Information Structure \ac{SOAP} over \ac{HTTP} sample request.},style=xml]{./gfx/addsecinfonosql.xml}

Registration of secondary information structure meta-data applies only for \ac{NoSQL} databases. The message described in Listing \ref{lst:testaddsis} contains the registration meta-data for an item stored in a table in the Amazon DynamoDB infrastructure. 

After the migrated data's meta-data is registered in the system, the tenant operator can access transparently the system using the standardized database access protocols, e.g. MySQL for the MySQL database system, and \ac{HTTP} for \ac{NoSQL} databases. 

\lstinputlisting[label={lst:testjdbc},caption={[Test Retrieving Information from Backend and Local Data Store]Retrieve data via \ac{JDBC} and MySQL protocol migrated to a backend Cloud data store, or local database system.},style=xml]{./gfx/testclient.java}

In Listing \ref{lst:testjdbc} we provide part of a Java program which connects via the MySQL Connector/J native driver with the MySQL CDASMix Proxy component on the port 3311. The options attached to the connection URL must be set as in the provided code in order to increase the performance of the communication, e.g. multi-querying for multiple queries in one statement, cache server configuration in order not to request the server configuration data per request, etc. In the MySQL request in Listing \ref{lst:testjdbc}, the tenant operator authenticates with the tenant and user UUID, and its password. Password is read in the system, but it is not compared in the authentication process, as the CDASMix MySQL proxy does not implement the hashing mechanisms of a MySQL server. After successful authentication, it executes a query which contains multiple queries in the same request. Each query is processed independently in the system when retrieving information from the different backend database systems, e.g. the \term{customer} table is stored in Amazon RDS, while mainInfoTest4 and mainInfoTest1 are stored in the local MySQL database system. The response is received as a single statement which contains n result sets, one for each query in the multi-query. 

\FloatBarrier
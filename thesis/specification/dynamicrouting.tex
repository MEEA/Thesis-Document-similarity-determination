\section{Dynamic Transparent Routing}
\label{sec:dynamicrouting}

Data access and modification from different back-end data stores must be supported in a transparent way between the tenants. We divide this requirement into two sub-requirements we consider our system must fulfill: Transparency, and Dynamic Routing.

\subsection{Transparency}

Giving a single logical view on a distributed database system abstracts the tenant from knowing the physical location of his data. The system must provide a single access endpoint for accessing and modifying data previously migrated to the Cloud. We must perform internally the necessary search operations in the registry containing the back-end data stores information, and the necessary mapping operations with the meta-data included in the tenants' requests. After those operations, our system must forward the requests to the back-end data stores and forward the responses to the tenants' requests.

\subsection{Dynamic Routing}
%dynamic between endpoitns and transformer
% dynamic between datasources, but no joins between different datasources
% dynamic between datasource with multi-querying option

Providing transparency by exposing a single endpoint to retrieve data from different back-end data stores requires the system to support dynamism in its routing mechanisms. One tenant may migrate one database to the Cloud, or \term{shard} a database between different databases in the Cloud, or different Cloud providers. This fact forces us to support connections to the back-end data stores dynamically rather than statically. Furthermore, when query and data transformation is required due to version or schema direct incompatibility between the tenant's requests and the back-end data store support, transformation mechanisms must ensure transformation of queries and data in order to provide a full transparent access. When transformations are not needed, the message should not be routed through a transformer component. In the prototype developed in this diploma thesis query and data transformations are out of scope.

When \term{sharding} a database, the tenant can split the data between two or more databases, or database systems. In order to minimize the changes in the application's DAL, we must support a single physical frontend endpoint while processing the tenant's request through one or more physical back-end endpoints. Special cases such as queries containing JOIN operations in between tables stored in different back-end \ac{SQL} databases are not supported. However, the system should support the execution of multiple \ac{SQL} queries in one request, which is known as \term{multiple-queries}.

\FloatBarrier
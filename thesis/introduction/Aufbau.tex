\section{Aufbau der Arbeit}
\label{sec:Aufbau der Arbeit}

%The remainder of this document is structured as follows:

%\begin{itemize}
%\item \textbf{Fundamentals, Chapter \ref{chap:fundamentals}:} provides the necessary background on the different concepts, technologies, and prototypes used in this diploma thesis.
%\item \textbf{Related Works, Chapter \ref{chap:relatedworks}:} discusses relevant State of the Art and positions our work towards it.  
%\item \textbf{Concept and Specification, Chapter \ref{chap:spec}:} functional and non-functional requirements are discussed in this section.
%\item \textbf{Design, Chapter \ref{chap:design}:} gives a detailed overview of the differerent component's architecture, and the needed extensions to the already existing ones.
%\item \textbf{Implementation, Chapter \ref{chap:implementation}:} the implemented components, as well as the necessary extensions or changes are detailed in this section from the point of view of coding and configuration. 
%\item \textbf{Validation and Evaluation, Chapter %\ref{chap:validationevaluation}:} in this chapter we test the final prototype based on the scenario described in this document. 
%\item \textbf{Outcome and Future Work, Chapter \ref{chap:outcome}:} we provide a conclusion of the developed work and investigate some ideas in which this diploma thesis can be extended.
%\end{itemize}
Die vorliegende Arbeit besteht aus sieben Kapiteln. Das Kapitel 1 befasst sich mit der Einleitung zum Thema dieser Arbeit, wo die Problemstellung, Zielsetzung sowie der Aufbau der Arbeit erläutert werden.
 
Das Kapitel \ref{chap:Grundlagen} behandelt die Grundlagen, die Analyse von Use Cases zur Dokumentähnlichkeitsbestimmung sowie auf Bayessche Statistik basierende Ansätze zur Dokumentähnlichkeitsbestimmung. In Kapitel \ref{sec:Informationsrückgewinnung} werden Grundlagen der Informationsrückgewinnung gegeben, nämlich die Definition des Begriffs Informationsrückgewinnung. Weitere wichtige Aspekte der Informationsrückgewinnung wie ihr Nutzen, ihr Mechanismus, ihre Modelle sowie ihre Web Variante werden erläutert. Kapitel \ref{sec:BayesscheStatistik} erläutert die Grundlagen der Bayessschen Statistik mit unteren Punkte über die Definition, der Hintergrund, die Anwendungsgebiete, der Nutzen, Modelle, Parameter, Überzeugungen der Bayessche Statistik, sowie die Wahrscheinlichkeit und der Satz von Bayes. Das Kapitel \ref{sec:AnalyseUse} beschäftigt sich mit der Analyse von Use Cases für die Dokumentähnlichkeitsbestimmung. Beim Kapitel \ref{sec:AnalyseAnsätze} wird die Analyse existierender Ansätze zur Dokumentähnlichkeitsbestimmung basierend auf Bayesschen Statistik.

Das Kapitel \ref{chap:BD4B} stellt die Informationsrückgewinnung-Middlewäre-Lösung vor. Dabei wird ihr Nutzen, Ihre Anforderungen, Fähigkeiten und Funktionsweise erläutert.

Im Kapitel \ref{chap:AuswahlAnsätzen} werden Ansätzen zur Dokumentähnlichkeitsbestimmung in der Informationsrück-
gewinnung-Middlewäre-Lösung ausgewählt. 

Im Kapitel \ref{sec:KriterienderAuswahl} werden Auswahlkriterien für Ansätzen zur Dokumentähnlichkeitsbestimmung erläutert und in Kapitel \ref{sec:FunktionsweiseAnsatz} wird die Funktionsweise des ausgewählten Ansatzes erläutert.

Im Kapitel \ref{chap:Implementierung} wird die Implementierung einer Teilmenge ausgewählten Ansätze erläutert.

Im Kapitel \ref{chap:Evaluierung} wird die Skalierbarkeit der realisierten Ansätze evaluiert durch die Benutzung von ausgewählten Probedaten. Das Kapitel  \ref{sec:FGrenzen} erläutert die Fähigkeiten und Grenzen dieser Ansätze.
 
Die vorliegende Arbeit wird mit dem Kapitel \ref{chap:zusammenfassung} durch eine Zusammenfassung und einen Ausblick abgeschlossen.

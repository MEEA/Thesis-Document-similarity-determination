\section{Zielsetzung}
\label{sec:Zielsetzung}

%In the following section we list the definitions and the abbreviations used in this diploma thesis for understanding the description of the work.

%\subsection*{Definitions}

%\subsection*{List of Abbreviations}

%The following list contains abbreviations which are used in this document. 

%\begin{acronym}[ApacheODEX]
%\acro{ACID}{Atomicity, Consistency, Isolation, Durability}
             % \acro{API}{Application Programming Interface}
%	\acro{ASP}{Application Service Provider}
%	\acro{Axis2}{Apache eXtensible Interaction System v. 2}
%	\acro{BC}{Binding Component}
%	\acro{BLOB}{Binary Large Object}
%		\acro{BPEL}{Business Process Execution Language 2.0}
%	\acro{C-CAST CMF}{Project Context Casting (C-CAST) Context-Management Framework}
%		\acro{CLI}{Command-line Interface}
%	\acro{CLOB}{Character Large Object}
%	    \acro{CORBA}{Common Object Request Broker Architecture}
%	         \acro{DBaaS}{Database-as-a-Service}
%	           \acro{DBMS}{Database Management System}
%	             \acro{DBS}{Database System}
%	                  \acro{DCOM}{Distributed Component Object Model}
%	                      \acro{DCE}{Distributed Computing Environment}
%	\acro{EAI}{Enterprise Application Integration}
%	\acro{EAR}{Enterprise Archive}
%	\acro{EJB}{Enteprise JavaBeans}
%	             \acro{EIP}{Enterprise Integration Patterns}
%	\acro{ESB}{Enterprise Service Bus}
%	        \acro{GAE}{Google App Engine}
%	            \acro{HTTP}{Hypertext Transfer Protocol}
%	\acro{IaaS}{Infrastructure-as-a-Service}
%	\acro{IDE}{Integrated Development Environment}
%	\acro{JAR}{Java Archive}
%	\acro{Java EE 5}{Java Platform, Enterprise Edition v. 5}
%	      \acro{J2EE}{Java 2 Platform, Enterprise Edition}
%    \acro{JAX-WS}{Java API for XML-Based Web Services}
%    \acro{JAXB}{Java Architecture for XML Binding}
%	\acro{JBI}{Java Business Integration}
%	\acro{JBIMulti2}{JBI Multi-tenancy Multi-container Support}
%	\acro{JDBC}{Java Database Connectivity}
%	\acro{JDK}{Java Development Kit}
%	\acro{JMS}{Java Message Service}
%	\acro{JMX}{Java Management Extensions}
%	         \acro{JNDI}{Java Naming and Directory Interface} 
%	\acro{JOnAS}{Java Open Application Server}
%	\acro{JPA}{Java Persistence API}
%	             \acro{JSON}{JavaScript Object Notation}
%	\acro{JSF}{JavaServer Faces}
%	\acro{JVM}{Java Virtual Machine}
%	        \acro{LRU}{Least Recently Used} 
%	\acro{MBean}{Managed Bean}
%		\acro{MDBS}{Multidatabase System}
%		       \acro{MEP}{Message Exchange Patterns}
%		             \acro{MOM}{Message-Oriented Middleware}
%	\acro{NIST}{National Institute of Standards and Technology}
%	       \acro{NM}{Normalized Message}
%	        \acro{NMF}{Normalized Message Format}
%	\acro{NMR}{Normalized Message Router}
%	               \acro{NoSQL}{Not only Structured Query Language}
%	\acrodep{OSGi}{Open Services Gateway initiative}
%	             \acro{ORDBMS}{Object-relational Database Management System}
%	\acro{PaaS}{Platform-as-a-Service}
%	\acro{POJO}{Plain Old Java Object}
%	        \acro{POM}{Project Object Model}
%	\acro{QoS}{Quality of Service}
%	             \acro{RDBMS}{Relational Database Management System}
%	                    \acro{SA}{Service Assembly}
%	\acro{SaaS}{Software-as-a-Service}
 %   \acro{SE}{Service Engine}
  %          \acro{SMPP}{Short Message Peer-to-Peer}
 %           \acro{SNMP}{Simple Network Management Protocols}
%	\acro{SOA}{Service-Oriented Architecture}
%	\acrodep{SOAP}{Simple Object Access Protocol}
%	              \acro{SQL}{Structured Query Language}
%	              	          \acro{STaaS}{Storage-as-a-Service}
%		                 \acro{SU}{Service Unit}
%		\acro{TCP}{Transmission Control Protocol}
%		       \acro{URI}{Uniform Resource Identifier}
  %  \acro{UUID}{Universally Unique Identifier}
  %  \acro{W3C}{World Wide Web Consortium}
  %          \acro{WAR}{Web Application Archive} 
  %  \acro{WS*}{Web Services (Specifications)}\acused{WS*}
   % \acro{WSDL}{Web Services Description Language}
  %  \acro{WSS4J}{Apache Web Services Security for Java}
   % \acro{XML}{eXtensible Markup Language}
  %  \acro{XSD}{XML Schema Definition}
  %  \acro{XSLT}{Extensible Stylesheet Language Transformation}
             

            

%\end{acronym}

Das Ziel dieser Masterarbeit ist es, die Dokumentähnlichkeit zu bestimmen für eine Big Data Informationsrückgewinnungslösung auf Basis von Ansätze, die auf Bayesschen Statistik basieren. Für die zuvor genannten Herausforderungen der Informationsrückgewinnungslösung, sollten in der vorliegenden Arbeit unterschiedliche Ansätze untersucht werden.  Die in dieser Arbeit Dokumentähnlichkeitsbestimmung soll so erfolgen, dass die oben genannte Probleme gelöst werden. 

Die Bayessche Statistik bietet Ansätze zur Dokumentähnlichkeitsbestimmung im Bereich der Informationsrückgewinnung, die angewendet und adaptiert zu gegebenen Fälle werden. Die Big Data Informationsrückgewinnungslösung kann zudem unterschiedlichen Datenquellen benutzen. Für die vorliegende Arbeit wird als Datenquelle hier ein Datenspeicher benutzt. Aus diesem Grund gilt es Ansätze zur Dokumentähnlichkeitsbestimmung basierend auf Bayessche Statistik zu analysieren Nach der Auswahl optimaler Ansätze wird eine Teilmenge der ausgewählten Ansätze implementiert. So können die realisierten Ansätze evaluiert werden auf Skalierbarkeit, das komplette auf Basis von ausgewählten Stichprobendaten

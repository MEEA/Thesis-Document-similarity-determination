\section{Motivation}
\label{sec:Motivation}     

A multi-tenant aware architecture in a Cloud environment is one of the main keys for profiting in a Cloud infrastructure. Virtualization and simultaneously usage of resources by multiple users allow Cloud providers to maximize their resources utilization. However, a multi-tenant environment requires isolation between the different users at different levels: computation, storage, and communication \cite{EnablingMT}. Furthermore, the communication to and from the Cloud infrastructure must support different protocols. 

Das Problem habe ich im Abschnitt~\ref{sec:problemstatement} beschrieben.

Migration of an application to the Cloud can be divided into four different migration types: component replacement with Cloud offerings, partial migration of functionalities, migration of the whole software stack of the application, and cloudifying the application \cite{andrikopoulos2013}. In this diploma thesis we focus on the needed support when the first migration type takes place. For example, due to an explosive data growth a tenant may decide at some point in time to migrate and host his local business data in a Cloud storage infrastructure, while maintaining his application's business logic on-premise. Bachmann provides a prototype which assists the tenant during the data migration process from a local storage system to a Cloud data store, and between Cloud data stores \cite{bachmann2012}. However, as described before his work covers the migration process, but it does not provide data access or data modification after the migration. 

An Enterprise Service Bus is a central piece in a \ac{PaaS} environment for providing flexible and loosely coupled integration of services as well as multi-tenant aware and multi-protocol communication between services. In this diploma thesis we extend the multi-tenant aware prototype of an \ac{ESB} produced in \cite{Muhler2012}, \cite{Essl2011}, and \cite{gomez2012}. The final prototype must provide multi-tenant and multi-protocol communication support, and transparent Cloud data access to tenants who migrate their application data partially or completely to the Cloud. 

The use of an intermediate component in data transfer may have a negative impact on the overall data processing in an application. For this reason, we provide an evaluation using example data from an existing TPC benchmark in order to investigate the impact on the performance and to propose future optimizations \cite{tcpbenchmark}.
\section{Motivation}
\label{sec:Motivation}     

%A multi-tenant aware architecture in a Cloud environment is one of the main keys for profiting in a Cloud infrastructure. Virtualization and simultaneously usage of resources by multiple users allow Cloud providers to maximize their resources utilization. However, a multi-tenant environment requires isolation between the different users at different levels: computation, storage, and communication \cite{EnablingMT}. Furthermore, the communication to and from the Cloud infrastructure must support different protocols. 

%Das Problem habe ich im Abschnitt~\ref{sec:problemstatement} beschrieben.

%Migration of an application to the Cloud can be divided into four different migration types: component replacement with Cloud offerings, partial migration of functionalities, migration of the whole software stack of the application, and cloudifying the application \cite{andrikopoulos2013}. In this diploma thesis we focus on the needed support when the first migration type takes place. For example, due to an explosive data growth a tenant may decide at some point in time to migrate and host his local business data in a Cloud storage infrastructure, while maintaining his application's business logic on-premise. Bachmann provides a prototype which assists the tenant during the data migration process from a local storage system to a Cloud data store, and between Cloud data stores \cite{bachmann2012}. However, as described before his work covers the migration process, but it does not provide data access or data modification after the migration. 

%An Enterprise Service Bus is a central piece in a \ac{PaaS} environment for providing flexible and loosely coupled integration of services as well as multi-tenant aware and multi-protocol communication between services. In this diploma thesis we extend the multi-tenant aware prototype of an \ac{ESB} produced in \cite{Muhler2012}, \cite{Essl2011}, and \cite{gomez2012}. The final prototype must provide multi-tenant and multi-protocol communication support, and transparent Cloud data access to tenants who migrate their application data partially or completely to the Cloud. 

%The use of an intermediate component in data transfer may have a negative impact on the overall data processing in an application. For this reason, we provide an evaluation using example data from an existing TPC benchmark in order to investigate the impact on the performance and to propose future optimizations \cite{tcpbenchmark}.

Im unternehmerischen Umfeld werden immer mehr großen Daten in einer oder mehreren Datenbanken gespeichert, verarbeitet und später ausgewertet, abhängig von dem verbundenen Zweck. Plattner behauptet in \cite{ENZ17}, dass Big Data ein Synonym für die Bedeutung großer Datenvolumen in verschiedensten Anwendungsbereichen sowie der damit verbundenen Herausforderung diese verarbeiten zu können, ist. So zitieren Fasel und Meier nach \cite{MER11}, dass Big Data definiert wird als Daten, die in ihrer Größe klassische Datenhaltung, Verarbeitung und Analyse auf konventioneller Hardware übersteigen \cite{FAM16}. Es besteht eine im Big Data eine Vielfalt an Daten, die sich voneinander unterscheiden. Mit dem immer wachsenden Datenvolum kann das Finden von bestimmten Daten jedoch mühsam und zeitaufwendig sein. 

Die Speicherung von großen Datenvolumen in relationalen Datenbanken kann Schwierigkeiten bereiten, bzw. wenn diese auf mehreren physischen Maschinen erfolgt. Es werden NoSQL-Technologien verwendet bei Unternehmen, die webbasiert sind. Die flexible Gestaltung und schnelle Änderung der Datenformate ist außerdem auch möglich durch die Nutzung von externe Quellen wie Webservices.

Durch digitale Transformation wollen sich Unternehmen entwickeln. Diese Entwicklung läuft über die Speicherung immer großer Datenmenge (Datensee). Jedoch erleichtert nicht dieser Datensee die Analyse und Informationsgewinnung. Die traditionelle Suche und unscharfe Informationsrückgewinnung bieten sich an um dieses Problem zu lösen. Diese sind aber nicht optimal, da sie nur als Basis vorhandenes Wissen haben. Eine Bremse zu der digitalen Transformation von Unternehmen ist das Vorhandensein eines Haufens von unbekannten Informationen, die sich qualitativ und auf Relevanz unterscheiden. Die Strukturierung der Information nach Inhalt und Bedeutung stellt sich als notwendig. Daher kann eine Softwarelösung, die Daten aus unterschiedlichen Datenbanken auf Inhalt vergleicht, eingestellt werden.

Eine Information Retrieval Middleware-Lösung deren Namen BigData4Biz ist, wird von der Firma Dibuco GmbH entwickelt. Diese Softwarelösung beruht auf einem Konzept der mehrdimensionalen Strukturierung eines Datensees unter Verwendung verschiedener Begriffe der Datenähnlichkeit.  Wobei die Datenähnlichkeit entweder geschäftsbezogen oder Geschäft agnostisch sein kann und beruht auf linguistische Aspekte, die berechnet werden unter anderem mit Berücksichtigung der Begriffsfrequenz und -bedeutung. Die Berücksichtigung der Begriffsfrequenz und -bedeutung hilft dabei Informationen im Datensee zu gewinnen und zu entdecken.

Eine Programmierschnittstelle ermöglicht es Abfragen in einer GUI zu formulieren um die Informationen zu gewinnen und zu entdecken. Angesichts der großen Menge an Daten eine Datensuche mit Benutzung von Sätze und Schlüsselwörter für eine Abfrage ist unmöglich. Unter Annahme, dass die Benutzer ganz unbewusst sind, was die Anzahl und die Art der Informationen und Daten angeht, BD4B bittet die Option für Benutzer bekannte Informationen zu sehen und abzufragen, sodass alternative und ähnliche Informationen als Ergebnisse angezeigt werden, so nach Relevanzgrad und mit einer immer engen Einschränkung bei der Suche.  

Im Kontext von großen Datenmengen, die immer mehr wachsen, ist die Skalierbarkeit sehr wichtig. Die vorliegende Arbeit setzt sich an diesem Punkt an und untersucht sowohl die unterschiedlichen Ansätze zur Ähnlichkeitsbestimmung, die auf Bayesschen Statistik basieren als auch Ihre Skalierbarkeit. Ziel ist es der bestmögliche Ansatz zur Dokumentähnlichkeit zu finden um eine optimale und adaptierte Datensuche in BigData4Biz zu ermöglichen. 





\chapter{Outcome and Future Work}
\label{chap:outcome}
%What have we done??
% we have evaluated the different technologies that were used and the different components which were separately implemented. multi-protocol multi-tenant communication in servicemix (soap http, jms, email and camel), included authentication in the esb, confidentiality and integrity is still missing. integrated version with the taxi scenario and separate endpoints configuration for both testing scenarios that we run, one for the taxi scenario and one for testing the endpoints individually. two echo services for testing the implemented approaches and used afterwords for the esb performance analysis. an analysis of the performance of the esb solution we are using and we have compared the penalty in the performance because of our modifications. we have also compared the actual performance improvement of distributing the load between two instances of servicemix and realized that is not a 50 % improved, this leads to a higher cost and less proportional improvement.  

Migration of one or more application layers to the Cloud aims to reduce the cost in the required IT infrastructure to host and maintain such layer within an organization. Adaptations on both migrated and non migrated layers are a must when part of an application is migrated to a Cloud infrastructure. In this diploma thesis we start from the basis of a partial migration of the application, particularly the database layer. The database layer includes the operations which provide data storage and retrieval support. Furthermore, the software, hardware and maintenance required to host and maintain this layer require an economical budget substantially greater than the needed for the business, or presentation layers of an application. Migrating the application's data to the Cloud requires rewiring the database connections to the backend Cloud data store, and adaptating the upper layers to match the operations, and data representations supported. Providing transparent communication support for accessing and storing data on application's databases in the Cloud demands a multi-protocol, and multi-tenant component. In this diploma thesis we extend a multi-tenant aware \ac{ESB} in order to utilize it as the database layer of the application, and access multiple Cloud data stores providing \ac{SQL} and \ac{NoSQL} databases. 

In Chapter \ref{chap:fundamentals} we present the necessary background about the technologies we use, and the components we reuse in this diploma thesis, e.g. JBIMulti2 \cite{Muhler2012}, \ac{JBI} and \ac{OSGi} frameworks, etc. Furthermore, we categorize the databases systems which are supported in the prototype, and subcategorize them based on their storage model and communication protocols. 

After researching on the \ac{SQL} and \ac{NoSQL} database systems properties in Chapters \ref{chap:fundamentals} and \ref{chap:relatedworks}, we find that most of database communication protocols are not standardized, and differ along the different database vendors. Therefore, we are forced in Chapter \ref{chap:design} to develop components which adjust to specific communication protocols: \ac{HTTP}, and MySQL. The research described in Chapter \ref{chap:relatedworks} leads us to find a lack of standardization in the communication at the TCP level of the \ac{SQL} database vendors, but the existence of components which support other communication protocols for incoming requests, e.g. \ac{HTTP}, and utilize via \ac{JDBC} the different database vendors' native driver to forward them to the backend database system \cite{jboss2011}. However, approaches in this direction forces the developer to adjust their data access layer, whose adaptations we aim to minimize when utilizing our system. \ac{NoSQL} Cloud data store providers show a lack of standardized naming, and provide the database users with their drivers for I/O operations. In order to address the lack of a standardized naming and access in the \ac{NoSQL} providers, we categorize in the system's registry the user's databases meta-data into different categories and subcategories, and access the backend data stores via the \ac{HTTP} protocol supported in ServiceMix-http-mt \cite{gomez2012}. 

The functional and non-functional requirements the system must fulfill are described in Chapter \ref{chap:spec}. After analyzing the requirements, providing an overview of the system, and specifying the necessary use cases, we move to the design of the prototype in Chapter \ref{chap:design}. We divide the design into the design of common components, and database specific components for the following databases types: \ac{SQL}, and \ac{NoSQL} databases. Apache ServiceMix-mt is used as the main component in the system for enabling a multi-protocol and multi-tenant communication support between backend databases. We design two different \ac{NMF}s' content for requests for enabling a dynamic routing of requests between the backend database systems. We extend the JBIMulti2 registries schemas to support the storage of tenant's migrated database configuration data. Components which implement the different databases systems communication protocols, e.g. MySQL and \ac{HTTP}, and components enabling routing between the multi-tenant aware consumer and provider endpoints are presented in Chapter \ref{chap:design}. However, the \ac{SQL} database support is limited in the system to one specific database system: MySQL. Separate components can be implemented to provide support for more \ac{SQL} database vendors, e.g. PostgreSQL or Oracle. Furthermore, the \ac{NoSQL} databases support is limited to the backend databases which support the \ac{HTTP} communication protocol. 

The implementation, validation, and evaluation of the system which complies with the requirements and the design of Chapters \ref{chap:spec} and \ref{chap:design} respectively, is explained in Chapters \ref{chap:implementation} and \ref{chap:validationevaluation}. We first validate the system by creating backend databases which contain custom data, and data generated by the TPC-H benchmark \cite{tcpbenchmark}. After configuring the communication configuration in CDASMix through the JBIMulti2 Web service interface, the tenant can communicate with the backend Cloud data store. Communication configuration should be done in the future through a user friendly Web interface, by integrating the \term{Cloud Data Migration Tool} and JBIMulti2 Web interfaces. 

For evaluating the advantages and disadvantages of utilizing CDASMix as the communication component in a database layer, we create an evaluation baseline for a backend MySQL database, run the different evaluation scenarios, and discuss its results. Applications with a high number of I/O operations often suffer a high performance decrease \cite{cashing2012}. Therefore, we enhance our prototype with a temporal multi-tenant cashing mechanism. Results describing the behavior of the system with a high load of data requests is described in Chapter \ref{chap:validationevaluation}, and demonstrate the advantages of cashing when accessing databases in the Cloud through CDASMix. An evaluation of the system's communication performance between \ac{NoSQL} databases is recommended in future works.

Further future works involve a secure authentication mechanism in CDASMix, as well as horizontal scalability of the system, and query transformation. The former involves implementing in the CDASMix MySQL Proxy the authentication mechanisms supported in the MySQL database server, and including password verification in the authentication phase in the multi-tenant \ac{HTTP} \ac{BC}. Horizontal scalability can be obtained between multiple instances of ServiceMix-mt building the system. Future versions of CDASMix can provide separate but connected ServiceMix-mt instances for routing requests for \ac{SQL} and \ac{NoSQL} database systems, or implementing a load balancer between the multiple instances building the data access layer. The developed version of CDASMix does not provide query and data transformation between different database versions, or database vendors. However, the system's design and implementation is extensible. A transformation component can be inserted between the endpoints in ServiceMix-mt.



%future work
%web gui, for both jbimulti2 and cloud data migration tool.-
% connection of the cloud data migration tool with the jbimulti2-
% benchmarking of the nosql
% benchmarking of the sql for more providers, and trying different configurations of database location and products and providers
% authentication in the system with the password
% support for incoming postgresql
% multiple instances of an esb connected to each other, and separate for example, one esb for sql, and one esb for nosql
% transformation of queries, and data

%The utilization of an \ac{ESB} as the main piece of middleware for \ac{SOA} in a Cloud environment forces multi-tenancy awareness to be a must in its requirements. This student thesis integrates the two main approaches for enabling multi-tenancy in an open source \ac{ESB}: multi-tenant aware messaging and multi-tenant aware administration and management, as well as analyzes and compares the performance of the native and extended \ac{ESB} solution in different scenarios, and produces as its main outcome an integrated version of the taxi application \cite{4CaaSt}. 

%In Chapter \ref{chap:fundamentals} we first provide the needed background on the technologies, communication protocols, and the main components this student thesis work with: ServiceMix and JBIMulti2 \cite{ASM}, \cite{Muhler2012}. After acquiring the main knowledge of the solutions, we investigate in Chapter \ref{chap:relatedworks} different solutions which support multi-tenancy, and analyze approaches which have been already taken into account. Furthermore, we discuss the supported functionalities of the AndoitLogic load generator driver, and the possibility of its reuse in our performance analysis. The identification of requirements and the system overview presented in Chapter \ref{chap:spec} guide us to perform the design of the different components of this student thesis in Chapter \ref{chap:design}. The design leads to a multi-tenant and multi-protocol aware version of ServiceMix, supporting three communication protocols: \ac{SOAP} over \ac{HTTP}, \ac{JMS}, and E-mail. Furthermore, we provide an integration design for the taxi application v2.0 prototype and we reengineer most of the implemented communication approaches in order to improve the system's performance and to add new functionalities, e.g. tenant context data structure modification, tenant authentication, and tenant-aware isolated endpoints in the \ac{ESB}. One of the main requirements in a Cloud infrastructure is security. We implement tenant authentication but not tenant data integrity and confidentiality. The tenant context information sent to and from the \ac{ESB} must be encrypted in future versions of the \ac{ESB}.

%Chapter \ref{chap:implementation} describes the challenges and approaches we faced for both the integration with the taxi scenario and the extension of the different \ac{JBI} \ac{BC}s. As discussed in Chapter \ref{chap:implementation}, one of the main goals we have in the improved version of the taxi application is to maximize the \ac{ESB} usage between components and to integrate a multi-tenant aware \ac{ESB} with non multi-tenant aware components which build part of the taxi application, e.g. \ac{BPEL} processes under Orchestra, CMF and GoogleDirections components, etc. This contrast forced us to perform changes in some of the taxi application components in order to adapt multi-tenancy at the communication level. Future versions of the taxi application should support multi-tenancy awareness in its components, and we consider that a bidirectional connection between JBIMulti2 and the taxi companies Web interfaces should be set in order to retrieve the tenant context information. Furthermore, the only communication protocol which actually supports the taxi application is the \ac{SOAP} over \ac{HTTP}. We have extended both \ac{JMS} and Mail \ac{JBI} \ac{BC}s for supporting a multi-protocol communication between customers and taxi drivers in a future version of the taxi application. This aspect directed us to build two different testing environments, one for the taxi application and one for the individual testing of the extended \ac{BC}s described in Chapter \ref{chap:test}.

%For analytical purposes after implementation, we perform an evaluation of the performance in Chapter \ref{chap:performanceevaluation} of both native and extended versions of ServiceMix. This student thesis reuses and extends an existing \ac{SOAP} over \ac{HTTP} \ac{ESB} performance benchmark. We adapt the benchmark to support multi-tenancy and evaluate the obtained results from different scenarios. However, we could not perform this analysis on more that one communication protocol. In the future it would be interesting to run the same scenarios on the \ac{XML} over \ac{JMS} communication protocol. Those results can give the \ac{ESB} administrator a better output for offering the communication protocols which best execute in our ServiceMix version. Furthermore, we perform the evaluation of one important scenario in a Cloud infrastructure: dividing the load between more than one ServiceMix instance by emulating a load balancer. The results showed that the performance is not significantly increased with the increase of the number of endpoints, and this approach can be not worth its expenditure. However, as we discussed, we emulate load balancing. For more than one instance of \ac{ESB} a load balancer should be integrated to the extended system. 

%Finally, we have integrated a multi-tenant \ac{ESB} which connect different endpoints via different protocols, e.g. external consumers with external providers. However, data is nowadays the most important asset of any business \cite{CHONGA2006}. The offering of a data-as-a-service solution in a Cloud environment where data can accessed through \ac{SOA} mechanisms, ables the use of the \ac{ESB} as a data access layer. With this approach data can be accessed from everywhere just by communicating with the \ac{ESB} and without worrying about the underlying architecture, e.g. database vendor, connection drivers. 



% Things to put: security in communication, JMS performance testing, using the ESB as Data Access Layer, optimization in the system resources usage of the actual implementation, load balancing between two or more instances of the esb, integration of the taxi scenario to communicate with jms or email, jms management of dominik improve it to have a bidirectional pipeline between jbimulti2 and servicemix, as well as connecting jbi multi2 with the taxi application and transmitter with security mechanisms in order to make a direct deployment of bcs from there or to just send the tenant and user information


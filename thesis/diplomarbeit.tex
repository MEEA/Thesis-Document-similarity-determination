%%%%%%%%%%%%%%%%%%%%%%%%%%%%%%%%%%%%%%%%%%%%%%%%%%%%%%%%%%%%%%%%%%%%%%%%%%%
%
% Template für Diplomarbeiten 
% (im speziellen für Diplomarbeiten an der Universität Stuttgart)
%
% Copyright (c) 2005 by Tim Schönleber
%
% This template is free; you can redistribute it and/or modify
% it under the terms of the GNU General Public License as published by
% the Free Software Foundation; either version 2 of the License, or
% (at your option) any later version.
%
% This program is distributed in the hope that it will be useful,
% but WITHOUT ANY WARRANTY; without even the implied warranty of
% MERCHANTABILITY or FITNESS FOR A PARTICULAR PURPOSE. See the
% GNU General Public License for more details.
%
% You should have received a copy of the GNU General Public License
% along with this program; if not, write to the Free Software
% Foundation, Inc., 59 Temple Place, Suite 330, Boston, MA 02111-1307 USA
%
% Author: Tim Schönleber
% Datum: 11.04.2005
% Version: 1.0
%
%%%%%%%%%%%%%%%%%%%%%%%%%%%%%%%%%%%%%%%%%%%%%%%%%%%%%%%%%%%%%%%%%%%%%%%%%%%

\documentclass[paper=a4,       % Papiergröße A4
					 11pt,
					 BCOR0mm,  % Bindekorrektur
					 DIV10,    % Satzspiegel mit 10er-Teilung
					 automark, % lebende Kolumnentitel
					 twoside,
					 halfparskip,
					 bibtotoc,
					 headsepline,
					 normalheadings,
					 appendixprefix,
					 pagesize  % Seitengröße wird bei dvi und pdf richtig gesetzt
 ]{scrbook}

% gängige Standardpakete
\usepackage[english]{babel}
\usepackage{graphicx}
\usepackage{microtype}
\usepackage{url}
\usepackage{mparhack, fixltx2e, ellipsis} % JW: diverse Bugfixes
\usepackage{colortbl, longtable, tabularx, lscape} %JW: farbige Tabellen, umgebrochene Tabellen, Tabellen mit Skalierung auf Seitenbreite, Querformat
\usepackage[pdftex]{xcolor} % benötigt um Schriftfarbe für Todos oder Hinweise zu ändern JW: Option hinzugefügt
\usepackage[english]{varioref} %JW: Verweise wie "Abbildung 4 auf der gegenüberliegenden Seite"


%%%%%%%%%%%% Schriften %%%%%%%%%%%%%%%%%%%%%%%%%%%%%%%%%%%%%%%%%%%%%%%%%%%%%%%%%
\usepackage[T1]{fontenc}				% T1-Schriften verwenden
\usepackage[utf8]{inputenc}

%Schriftart = Palatino
\usepackage{mathpazo}
\usepackage[scaled=0.9]{helvet}
\usepackage[scaled=0.885]{luximono}		% TeleType-Schrift: Luxi Mono

\usepackage{setspace}					% 1.05-facher Zeilenabstand wegen Palatino
\linespread{1.05}
%%%%%%%%%%%% Schriften %%%%%%%%%%%%%%%%%%%%%%%%%%%%%%%%%%%%%%%%%%%%%%%%%%%%%%%%%
%Package A4wide, um den Platz auf einem A4 Papier besser auszunutzen   
\usepackage{a4wide} %JW: Dieses Paket widerspricht jeglicher Regel für schöne Typografie

%Package Booktabs für "`schönere"' Tabellen
\usepackage{booktabs}

%Paket Listings für "`schöne"' Code-Umgebungen
\usepackage{listings}

\usepackage{rotating}

\usepackage{placeins}

\usepackage{subfigure}

\usepackage[htt]{hyphenat}

\usepackage{array}

\usepackage[printonlyused]{acronym}

\usepackage[pass]{geometry}

\usepackage{mathtools}

%Paket Caption zur Formatierung von Tabellen- und Bildunterschriften 
\usepackage[margin=25pt,font=small,labelfont=bf, format=hang]{caption}

%Paket Hyperref für den pdf-Export
\usepackage[pdfpagelabels]{hyperref}
\hypersetup
	{%
	pdfsubject = {Bestimmung der Dokumentenähnlichkeit basierend auf Bayessche Statistik für eine Big-Data Information Retrieval Lösung},
	pdfauthor = {Elisabeth Agnes Mpessa Enangue},
	pdfkeywords = {},
	colorlinks = {true},
	citecolor = {black},
	linkcolor = {black},
	urlcolor  = {black},
	a4paper
	}

% Package ntheorem für Definitionen, Sätze, etc.
\usepackage[standard,hyperref]{ntheorem}
\newtheorem{Def}{Definition}[chapter]
\newtheorem{Begriff}{Term}[chapter]

%----------------  Kopf- und Fußzeilen  --------------------------------------
\usepackage{scrpage2}
\pagestyle{scrheadings}

\cfoot[]{}
\ifoot[]{}
\ofoot[\pagemark]{\pagemark}
\ohead[]{}
\chead[]{}
\ihead{\headmark}
%-----------------------------------------------------------------------------

%%%%%%%%%%%%%%%%%%%%%%%%%%%%%%%%%%%%%%%%%%%
%Änderung der Kapitelüberschriften
%%%%%%%%%%%%%%%%%%%%%%%%%%%%%%%%%%%%%%%%%%%

%-----------------------------------------------------------------------------
\renewcommand*{\chapterheadstartvskip}{}
%-----------------------------------------------------------------------------

\newcommand*{\ORIGchapterheadstartvskip}{}
\let\ORIGchapterheadstartvskip\chapterheadstartvskip
 \renewcommand*{\chapterheadstartvskip}{
   \ORIGchapterheadstartvskip
   {
     \setlength{\parskip}{0pt}
     \noindent\rule[.3\baselineskip]{\linewidth}{1pt}
   }
 }
\newcommand*{\ORIGchapterheadendvskip}{}
\let\ORIGchapterheadendvskip\chapterheadendvskip
\renewcommand*{\chapterheadendvskip}{
{
\setlength{\parskip}{0pt}
\noindent\rule[.3\baselineskip]{\linewidth}{1pt}\par
}
\ORIGchapterheadendvskip
}
%%%%%%%%%%%%%%%%%%%%%%%%%%%%%%%%%%%%%%%%%%%
%Ende Änderung der Kapitelüberschriften
%%%%%%%%%%%%%%%%%%%%%%%%%%%%%%%%%%%%%%%%%%%

%Trennungsliste
\hyphenation{}

%%%%%%% Custom Commands %%%%%%%%%%%%%%%%%%%%%%%%%%%%%%%%%%%%%%%%%%%%%%%%%%%%%%%%

\newcommand{\term}[1]{\textit{#1}}
\newcommand{\code}[1]{\texttt{#1}}
\newcommand{\refsection}[1]{(see~Sect.~\ref{#1})}
\newcommand{\refappendix}[1]{(see~Appendix~\ref{#1})}
\newcommand{\reflisting}[1]{(see~Listing~\ref{#1})}
\newcommand{\reffig}[1]{(see~Fig.~\ref{#1})}
\newcommand{\reftable}[1]{(see~Table~\ref{#1})}

\newcommand{\acrodep}[2]{\acro{#1}{#2 \acroextra{ \textit{(deprecated)}}}\acused{#1}}

%%%%%%% Listings %%%%%%%%%%%%%%%%%%%%%%%%%%%%%%%%%%%%%%%%%%%%%%%%%%%%%%%%%%%%%%%

%Setzen der Einstellungen für das Listings-Paket am Beispiel Java
%\definecolor{comments-eclipse-style}{rgb}{0.25,0.5,0.37}
%\definecolor{keywords-eclipse-style}{rgb}{0.5,0.0,0.33}
%\definecolor{strings-eclipse-style}{rgb}{0.16,0.0,1.0}

% JW: Beachte, dass jede farbige Seite Geld kostet oder im Ausdruck anders rüberkommt. Ich hab alles schwarz-weiß gemacht. Außerdem helfen Nummerierungen für Verweise. Kommentiere den Stil aus, der Dir mehr zusagt.
\definecolor{comments-eclipse-style}{rgb}{0.5,0.5,0.5}
\definecolor{keywords-eclipse-style}{rgb}{0.35,0.35,0.35}
\definecolor{strings-eclipse-style}{rgb}{0.0,0.0,0.0}

\newcommand\myKeywordStyle{\bfseries}

\definecolor{black}{rgb}{0.0,0.0,0.0}

\lstset{%
	basicstyle=\ttfamily,
	numbers=left,
	numberstyle=\tiny,
	stepnumber=1,
	numbersep=10pt,
	frame=tb,
	framerule=\heavyrulewidth,
	tabsize=2,
	captionpos=b,
	breaklines=true,
    breakatwhitespace=false,
    aboveskip=1.5em,
    belowskip=1em,
	keywordstyle=\color{keywords-eclipse-style},
	commentstyle=\color{comments-eclipse-style},
	stringstyle=\color{strings-eclipse-style}
}

%JW: Original
%\lstdefinestyle{java}
%	{language=Java,
%	 basicstyle=\ttfamily,
%	 identifierstyle=\ttfamily,
%	 keywordstyle=\color{keywords-eclipse-style},
%	 commentstyle=\color{comments-eclipse-style},
%	 stringstyle=\color{strings-eclipse-style},
%	 %numbers=left,
%	 %numberstyle=\tiny,
%	 %stepnumber=1,
%	 %numbersep=10pt,
%	 frame=single,
%	 captionpos=b
%}

%\lstdefinestyle{xml}
%	{language=XML,	 
%	 basicstyle=\ttfamily,
%	 identifierstyle=\ttfamily,
%	 keywordstyle=\color{keywords-eclipse-style},
%	 commentstyle=\color{comments-eclipse-style},
%	 stringstyle=\color{strings-eclipse-style},	 
%	 captionpos=b,
%	 showstringspaces=false
%}

%\lstdefinestyle{xml}
%	{language=XML,	 
%	 basicstyle=\ttfamily \small,
%	 captionpos=b,
%	 showstringspaces=false
%}
%
\lstdefinestyle{Java}
	{language=Java,	 	 
	 basicstyle=\ttfamily \footnotesize
}
\lstdefinestyle{xml}
	{language=XML,	 	 
	 basicstyle=\ttfamily \footnotesize
}
\lstdefinelanguage{EBNF} {
	morecomment=[s]{/*}{*/},
	morestring=[b]"
}
\lstdefinestyle{ebnf}
	{language=EBNF,	 	 
	 basicstyle=\ttfamily \footnotesize,
	 stringstyle=\color{keywords-eclipse-style}
}
%%%%%%% Ende Listings %%%%%%%%%%%%%%%%%%%%%%%%%%%%%%%%%%%%%%%%%%%%%%%%%%%%%%%%%%


%Chapter-Überschriften linksbündig
\renewcommand*{\chapterformat}{%
\chapappifchapterprefix{\ \hspace{0.75cm}}
\makebox[5pt][r]{\thechapter\autodot\enskip}}

%%% Intelligente Querverweise (Verwendung siehe meine Arbeit) %%%%%%%%%%%%%%%%%%
%% Copied from the TeXbook
\newcommand \jwIfundefined[1]{%
\expandafter\ifx\csname#1\endcsname\relax}

\newcommand \myRefRange[3]{%
\newcount\myCounter
\myCounter=#3
\advance \myCounter by -#2%
\advance \myCounter by -1 %
\ifnum \myCounter=0\relax
	\ref{#1} auf Seite~#2\,f.%
\else
	{\ref{#1} auf Seite~#2\ bis~\pageref{#1:end}}%
\fi
}

\newcommand \myRefi[1]{%
\jwIfundefined{r@#1:end}{\vref{#1}}\else
{% 3 Fälle:
%  - nur eine Seite: kein Bereich, sondern wie \vref
%  - 2 Seiten -> f
%  - > 2 Seiten: Bereich
\vrefpagenum\jwFirstnum{#1}%
\vrefpagenum\jwSecondnum{#1:end}%
%
\ifthenelse{\equal\jwFirstnum\jwSecondnum}%
{\vref{#1}}% Fall 1
{\myRefRange{#1}\jwFirstnum\jwSecondnum}% Secondnum >= firstnum + 1
}\fi
}

\newcommand\myVref[1]{%
\jwIfundefined{r@#1}{\vref{#1}}\else
{\myRefi{#1}}\fi
}
%%%%%%%%%%%%%%%%%%%%%%%%%%%%%%%%%%%%%%%%%%%%%%%%%%%%%%%%%%%%%%%%%%%%%%%%%%%%%%%%


\begin{document}

%%%%%%%%%%%%%%%%%%%%%%%%%%%%%%%%%%%%%%%%%%%
%Erzeugung des Titelblatts
%%%%%%%%%%%%%%%%%%%%%%%%%%%%%%%%%%%%%%%%%%%
% Deckblatt zentrieren
%\newlength\oddsidemarginorig
%\oddsidemarginorig=\oddsidemargin
%\oddsidemargin 1.05cm
%\newgeometry{a4paper,left=20.05mm,right=10mm, top=29.7mm, bottom=29.7mm}
%\thispagestyle{plain}
\pagestyle{plain}
\begin{titlepage}
\begin{sffamily}
\begin{center}
Fakult\"at Informatik\\
Hochschule Reutlingen\\
Alteburgstra\ss{}e 150\\
D-72762 Reutlingen\\
\end{center}

\vspace{3.5cm}

\begin{center}
{Masterthesis}\\
\vspace{0.5cm}
\begin{minipage}{8.5cm}
\begin{center}
 
  \Large \textbf{Bestimmung der Dokumenten\"ahnlichkeit basierend auf Bayessche Statistik f\"ur eine Big-Data Information Retrieval L\"osung}
 
\end{center}
\end{minipage}
\\
\vspace{1cm}
{Elisabeth Agnes Mpessa Enangue}
\end{center}

\vspace{1.0cm}

\begin{center}
\begin{minipage}{3cm}
\begin{center}
	\includegraphics[width=2\textwidth]{./gfx/logo_INFFH.png}
	
\end{center}
\end{minipage}
\begin{minipage}{3cm}
\begin{center}
%	\includegraphics{./gfx/iaas.jpg}
\end{center}
\end{minipage}
\end{center}
%
\vspace{1.0cm}
%
\begin{center}
\begin{tabular}{ll}
\textbf{Studiengang:} & Services Computing\\
&\\&\\
\textbf{Betreuer Hochschule:}   & Prof Dr.-Ing Christian Decker\\
\textbf{Betreuer Unternehmen:}   & Steve Strauch\\
                     
\textbf{Abgabetermin:} &   31. Juli 2018\\
%\textbf{Completed:}  &  March 22, 2013\\
%&\\
%\textbf{CR-Classification:} & C.2.4, C.4, D.2.11, H.2.0, H.2.4\\

\end{tabular}
\end{center}
\end{sffamily}
\end{titlepage}
%\oddsidemargin=\oddsidemarginorig
\restoregeometry
%%%%%%%%%%%%%%%%%%%%%%%%%%%%%%%%%%%%%%%%%%%
%Ende Titelblatt
%%%%%%%%%%%%%%%%%%%%%%%%%%%%%%%%%%%%%%%%%%%
\cleardoubleemptypage

\setlength{\parindent}{0.0em}

\pagenumbering{roman}

\begin{center}
\section*{Abstract}
\end{center}
\pagestyle{empty}

In the last years Cloud computing has become popular among IT organizations aiming to reduce its operational costs. Applications can be designed to be run on the Cloud, and utilize its technologies, or can be partially or totally migrated to the Cloud. The application's architecture contains three layers: presentation, business logic, and data layer. The presentation layer provides a user friendly interface, and acts as intermediary between the user and the application logic. The business logic separates the business logic from the underlaying layers of the application. The Data Layer (DL) abstracts the underlaying database storage system from the business layer. It is responsible for storing the application's data. The DL is divided into two sublayers: Data Access Layer (DAL), and Database Layer (DBL). The former provides the abstraction to the business layer of the database operations, while the latter is responsible for the data persistency, and manipulation. 

When migrating an application to the Cloud, it can be fully or partially migrated. Each application layer can be hosted using different Cloud deployment models. Possible Cloud deployment models are: Private Cloud, Public Cloud, Community Cloud, and Hybrid Cloud. In this diploma thesis we focus on the database layer, which is one of the most expensive layers to build and maintain in an IT infrastructure. Application data is typically moved to the Cloud because of , e. g. Cloud bursting, data analysis, or backup and archiving. Currently, there is little support and guidance how to enable appropriate data access to the Cloud. 

In this diploma thesis the we extend an Open Source Enterprise Service Bus to provide support for enabling transparent data access in the Cloud. After a research in the different protocols used by the Cloud providers to manage and store data, we design and implement the needed components in the Enterprise Service Bus to provide the user transparent access to his data previously migrated to the Cloud.

\cleardoubleemptypage
\pagestyle{scrheadings}

\tableofcontents
\newpage

\listoffigures
\listoftables
\renewcommand{\lstlistlistingname}{List of Listings}
\lstlistoflistings

\cleardoubleemptypage

%\sloppy

\pagenumbering{arabic}
\newpage
\chapter{Introduction}
\label{chap:introduction}
%\ac{PaaS}

Cloud computing has changed in the last years the computing resources consumption and delivery model in the IT industry, leading to offer them as services which can be can be accessed over the network. The idea of virtualizing resources and provide them on demand to the users, like water, electricity, etc. under a metered service aims to deliver computing as a utility. Users are offered a single system view in a fully virtualized environment where computing resources seem unlimited, and can be accessed through Web interfaces and standardized protocols. Cloud providers target to maximize their benefits by maximizing the resources usage with a minimum management effort or human interaction, while the Cloud consumers can significantly reduce their capital expenditures in their IT infrastructure by outsourcing the demanded computational and storage resources to a Cloud environment.

In the following sections we discuss the problem statement and motivating scenario this thesis relies on. 

\section{Problem Statement}
\label{sec:problemstatement}     

A multi-tenant aware architecture in a Cloud environment is one of the main keys for profiting in a Cloud infrastructure. Virtualization and simultaneously usage of resources by multiple users allow Cloud providers to maximize their resources utilization. However, a multi-tenant environment requires isolation between the different users at different levels: computation, storage, and communication \cite{EnablingMT}. Furthermore, the communication to and from the Cloud infrastructure must support different protocols. 

Migration of an application to the Cloud can be divided into four different migration types: component replacement with Cloud offerings, partial migration of functionalities, migration of the whole software stack of the application, and cloudifying the application \cite{andrikopoulos2013}. In this diploma thesis we focus on the needed support when the first migration type takes place. For example, due to an explosive data growth a tenant may decide at some point in time to migrate and host his local business data in a Cloud storage infrastructure, while maintaining his application's business logic on-premise. Bachmann provides a prototype which assists the tenant during the data migration process from a local storage system to a Cloud data store, and between Cloud data stores \cite{bachmann2012}. However, as described before his work covers the migration process, but it does not provide data access or data modification after the migration. 

An Enterprise Service Bus is a central piece in a \ac{PaaS} environment for providing flexible and loosely coupled integration of services as well as multi-tenant aware and multi-protocol communication between services. In this diploma thesis we extend the multi-tenant aware prototype of an \ac{ESB} produced in \cite{Muhler2012}, \cite{Essl2011}, and \cite{gomez2012}. The final prototype must provide multi-tenant and multi-protocol communication support, and transparent Cloud data access to tenants who migrate their application data partially or completely to the Cloud. 

The use of an intermediate component in data transfer may have a negative impact on the overall data processing in an application. For this reason, we provide an evaluation using example data from an existing TPC benchmark in order to investigate the impact on the performance and to propose future optimizations \cite{tcpbenchmark}.
\input{introduction/motivatingscenario}
\input{introduction/abbreviations}
\input{introduction/outline}
\chapter{Fundamentals}
\label{chap:fundamentals}

In this chapter we give an explanation about the technologies and concepts this diploma thesis relies on. We start describing the fundamental concepts and introduce the components and prototypes that form the basis of our work.

\input{fundamentals/cloudcomputing}
\input{fundamentals/soa}
\section{Multi-tenancy}
\label{sec:multitenancy}   

One of the main decision variables for utilizing a Cloud computing environment are capital expenditures. The main goal of a Cloud consumer is to minimize its business costs when migrating to the Cloud. According to Chong, a \ac{SaaS} solution benefits a Cloud customer with the following advantages \cite{ChongB2006}:

	\begin{itemize}
		\item The Cloud consumer does not directly purchase a software license, but a subscription to the software offered as a service by the Cloud infrastructure. 
		\item More than half of the IT investments of a company are made in infrastructure and its maintenance. In a \ac{SaaS} solution this responsibilities are mainly externalized to the Cloud provider.   
		\item A Cloud computing environment is based on the utilization of its resources simultaneously by a large number of Cloud consumers. For example, a Cloud provider that offers a centrally-hosted software service to a large number of customers can serve all of them in a consolidated environment and lower the customer software subscription costs while maintaining or lowering the provider's infrastructure, administration and maintenance costs. 
		\item The cost leverage in the software utilization allows the Cloud providers to focus not only on big enterprises capable of large IT budgets, but also on the small business that need access to IT solutions. 
	\end{itemize} 

Multi-tenancy in a \ac{SaaS} environment allows the Cloud providers to lower the cost per customer by optimizing the resources usage in the Cloud infrastructure. The software serves multiple tenants concurrently, who share the same code base and data storage systems. Chong and Carraro \cite{ChongB2006} define a well designed \ac{SaaS} application as scalable, multi-tenant-efficient and configurable. With this design patterns, the \ac{SaaS} model enables the provider to \term{catch the long tail}. Business softwares are becoming more complex and tend to demand an individual customer support and an increase of the computing and storage resources in the infrastructure. This fact leads to an increase in the infrastructure investment and maintenance costs. However, if the previous requirements are eliminated and the provider's infrastructure is scaled to combine and centralize customers' hardware and services requirements, the price reduction limit can be decreased and, in effect, allow a wide range of consumers to be able to access this services.

The reasons discussed above are also applicable in the \ac{DBaaS} and \ac{STaaS} models. Storage and retrieval of data involve high maintenance and management costs. The data management cost is estimated to be between 5 to 10 times higher than the data gain cost \cite{multishares2011}. Furthermore, storing data on-premise does not only require storing and retrieving data, but also requires dealing with disaster recovery, \ac{DBMS}, capacity planning, etc. Most of the organizations prefer to lead their investments to their local business applications rather than becoming a data management company \cite{multishares2011}. Cloud storage providers offer a pay-per-use storage model, e.g. based on storage capacity or based on number of connections to the storage system, and ensure that the stored data will persist over time and its access through the network. However, security and confidentiality are the main constraints when moving private data to a shared public infrastructure. 

\begin{figure}[htb]
	\centering
		\includegraphics[clip, scale=0.5]{./gfx/longtail.png}
	\caption[Multi-tenancy and Long Tail]{New market opened by lower cost of SaaS \cite{ChongB2006}}
	\label{fig:longtail}
\end{figure}

In the Figure \ref{fig:longtail} the economics of scaling up to a high number of customers while reducing the software price is analyzed. Cloud providers have reached a new market formed by small or medium businesses without enough budget for building an on-premise IT infrastructure.

Multi-tenancy refers to the sharing of the whole technological stack (hardware, operating system, middleware, and application instances) at the same time by different tenants and their corresponding users \cite{EnablingMT}. Andrikopoulos et al. identify two fundamental aspects in multi-tenant awareness: communication, and administration and management \cite{andrikopoulos2013}. The former involves isolated message exchanges between tenants and the latter allows tenants individual configuration and management of their communication endpoints. Utilizing an \ac{ESB} as the central piece of communication middleware between applications in a \ac{PaaS} environment forces it to ensure multi-tenancy at both communication, and administration and management, as mentioned before. The multi-tenancy support modifications made in the open-source ServiceMix 4.3 are the results of \cite{Essl2011}, \cite{Muhler2012}, and \cite{gomez2012}. In this diploma thesis we reuse and extend those results in oder to provide multi-tenant transparent Cloud data access in the Cloud through the \ac{ESB}, when the application's data is migrated and accessed through the \ac{ESB} in a Cloud infrastructure.

The migration of an application's stack to the Cloud can be done at different levels of the application's stack: Presentation Layer, Business Layer, and Data Access Layer. The Replacements of Components with Cloud offerings migration type is the least invasive type of migration \cite{andrikopoulos2013}. In this diploma thesis we focus on this type of migration, concretely when the used Cloud offering is the database system. Migration of the data can be either seen as the migration of the Data Layer (Data Access Layer and Database Layer) or of the whole application \cite{andrikopoulos2013}. Migration of the Data Layer to the Cloud means migrating the both data management and data access to the Cloud, while maintaining its transparency to the application's Business Layer. 

In a Cloud infrastructure where Cloud storage is offered, Feresten identifies four main tenant requirements: security, performance, data protection and availability, and data management \cite{feresten2010}. Multi-tenancy in a storage system can be achieved by aggregating tenant-aware meta-data to the tenant's data (see Figure \ref{fig:virtualstoragecontainer}), or by physical storage partitioning, but this is not sufficient when fulfilling the data management, and the flexibility requirement. Tenants must have independent access and management, as if they accessed their own data storage systems. For this purpose, storage vendors have introduced the concept of \term{virtual storage container}, a tenant-aware management domain which grants all (or most of) the database management operations over the storage container, as described in Figure \ref{fig:virtualstoragecontainer}.

\begin{figure}[htb]
	\centering
		\includegraphics[clip, scale=0.4]{./gfx/virtualstoragecontainer.png}
	\caption[Virtual Storage Container]{Attributes of a Virtual Storage Container \cite{feresten2010}}
	\label{fig:virtualstoragecontainer}
\end{figure}

In this diploma thesis we must take into account the different approaches that most of the Cloud storage vendors have taken into account, in order to provide the tenant transparent access through the \ac{ESB} to his virtual storage container in one or more Cloud infrastructures.
\input{fundamentals/jbi}
\input{fundamentals/osgi}
\input{fundamentals/servicemix}
\section{Binding Components}
\label{sec:bindingcomponents}  

In this section we describe the \ac{JBI} \ac{BC}s shipped in the ServiceMix-mt prototype this diploma thesis focuses on, and the transport protocols they support. 

\subsection{Multi-tenant HTTP Binding Component}

ServiceMix provides \ac{HTTP} communication support in its \ac{HTTP} \ac{JBI} \ac{BC}. Its \ac{HTTP} consumer and provider endpoints are built on the \ac{HTTP} Jetty 6 server and Jakarta Commons \ac{HTTP} Client respectively, providing support for both REST and SOAP over HTTP 1.1 and 1.2 requests.

The original \ac{HTTP} \ac{BC} is extended in the ServiceMix-mt prototype to provide multi-tenant support in \cite{Muhler2012} and \cite{gomez2012}. Muhler provides an internal dynamic creation of tenant-aware endpoints in the \ac{BC}, by injecting tenant context in the \ac{JBI} endpoint's URLs \cite{Muhler2012}. Gomez provides a \ac{NMF} with tenant context information in its properties for routing in the \ac{NMR} \cite{gomez2012}. However, in this diploma thesis we must not only provide tenant isolation at the tenant level, but also isolation at the user level. We discuss this requirement in detail in Chapters \ref{chap:spec} and \ref{chap:design}.

\begin{figure}[htb]
	\centering
		\includegraphics[clip, scale=0.3]{./gfx/httpmtbc.pdf}
	\caption[Multi-tenant HTTP Binding Component]{Multi-tenant HTTP Binding Component \cite{gomez2012}. }
	\label{fig:httpmt}
\end{figure}

As seen in Figure \ref{fig:httpmt}, the multi-tenant \ac{HTTP} \ac{BC} is mainly used in ServiceMix-mt to support the \ac{SOAP} over \ac{HTTP} communication protocol by exposing a Web service in the tenant-aware consumer endpoint and consuming an external Web service in the provider endpoint. \ac{SOAP} defines an \ac{XML} message format  which is sent over the network and a set of rules for processing the \ac{SOAP} message in the different \ac{SOAP} nodes which build the message path between two endpoints \cite{Weera2005}. A \ac{SOAP} message is a composition of three main elements: a SOAP envelope, header, and body. A SOAP envelope may contain zero or more headers and one body. The header may contain processing or authentication data for the ultimate receiver or for the intermediate nodes through the message is routed. The message payload or business data is included in the SOAP body. SOAP is used as a message framework for accessing Web services in loosely coupled infrastructures \cite{Weera2005}. The Web service consumer specifies the functionality to invoke in the SOAP body. If the Web service functionality has a request-response \ac{MEP}, a SOAP message is used to send the response data when the corresponding operation has been executed successfully or the error data in case an error occurred during execution.

Most of the Cloud storage providers provide an \ac{HTTP} interface to the tenants for data management, retrieval, and storage. In this diploma thesis we extend this \ac{JBI} \ac{BC} in order to provide the tenant a transparent access to his \ac{NoSQL} Cloud data stores.

\FloatBarrier

\section{Service Engine}
\label{sec:serviceengine}  

A \ac{SE} can provide different kinds of services, e.g. business logic, routing, and message transformation. In this diploma thesis we will mainly concentrate on one: Apache Camel \cite{Camel2011}, which is wrapped in a ServiceMix-camel \ac{JBI} \ac{SE} in ServiceMix, and in a ServiceMix-camel-mt \ac{JBI} \ac{SE} in ServiceMix-mt  for multi-tenancy awareness..

\subsection{Apache Camel}

Apache Camel is an open-source integration framework based on known \ac{EIP} which supports the creation of routes and mediation rules in either a Java based Domain Specific Language (or Fluent API), via Spring based XML Configuration files or via the Scala DSL \cite{Camel2011}. In ServiceMix, Apache Camel is shipped in a \ac{JBI} \ac{SE}. The routing or mediation rules between two or more endpoints can be specified in an Spring Configuration file or in a \ac{POJO} file whose's class extends the Apache Camel \term{RouteBuilder} class. Route configurations deployed in ServiceMix must follow the \ac{JBI} compliance: files describing the route configuration must be packed in \ac{SU}, and the latter in a \ac{SA}. Apache Camel provides Maven archetypes which generate the needed route configuration files (in \ac{XML} or \ac{POJO} formats) where the developer can program the route between the different supported endpoints \cite{MAVEN}. The configuration in a \ac{XML} file reduces the configuration complexity to a minimum effort of the developer. However, a configuration in a \ac{POJO} class increases the developing complexity but allows the developer to provide logic, filtering, dynamic routing, etc. In the \term{RouteBuilder} class a developer can access, for example, the header of a \ac{NM} and select the target endpoint dynamically depending on the implemented logic. Furthermore, the routing patterns supported by Apache Camel are the point-to-point routing and the publish/subscribe model. 

The endpoints representation in Apache Camel is based on \ac{URI}. This allows this \ac{SE}s to integrate with any messaging transport protocol supported in the \ac{ESB}, e.g. \ac{HTTP}, \ac{JMS} via ActiveMQ, E-Mail, CXF, etc. The ServiceMix-camel \ac{JBI} \ac{SE} provides integration support between camel and \ac{JBI} endpoints. Muhler extends this component and allows dynamic internal creation of tenant-aware endpoints in the ServiceMix-camel-mt \ac{JBI} \ac{SE} \cite{Muhler2012}. The main goal of this extension is to provide an integrated environment between \ac{JBI} and camel supported endpoints. However, multi-tenancy is supported at the tenant level only in the \ac{JBI} endpoints. In this diploma thesis we aim to enable multi-tenancy not only at the tenant level, but also at the user level, as discussed in Chapters \ref{chap:spec} and \ref{chap:design}.

For enabling data access support with \ac{SQL} \ac{DBMS} in ServiceMix-mt we extend a well-known camel component: Camel-jdbc. The Camel-jdbc component enables \ac{JDBC} access to \ac{SQL} databases, using the standard \ac{JDBC} API, where queries and operations are sent in the message body \cite{cameljdbc}. This component requires the developer to statically denote the data source configuration (user, password, database name, etc.) in both the endpoint \ac{URI} and route configuration file. As discussed in Chapters \ref{chap:spec} and \ref{chap:design}, this requirement is opposite to our approach, due to the dynamism we need in creating connections to the different \ac{DBaaS} providers. We extend and produce a custom camel component: Camel-jdbccdasmix (\term{cdasmix} stands for Cloud Data Access Support in ServiceMix-mt).

\FloatBarrier

\section{Structured Query Language Databases}
\label{sec:fundamentalssql}  

The \ac{SQL} stands nowadays as the standard computer database language in \ac{SQL} \ac{DBS}. \ac{SQL} is a tool for organizing, managing, and retrieving data stored by a computer database \cite{sql1999}. The \ac{SQL} language is nowadays one of well known languages in the IT sector. Thus, we introduce in the following sections in the \ac{SQL} \ac{DBS} and their specific communication protocol we use in this thesis, rather than on the language they support. 

The final prototype of this diploma thesis aims to provide support for most of the \ac{DBS} communication protocols available in the market. However, we crashed into vendor-specific communication protocol implementations along the available \ac{DBMS} in the market, rather than a common standardized communication protocol. For this reason, we provide support for incoming connections which comply the MySQL \ac{DBS} communication protocol and give the hints for supporting the PostgreSQL \ac{DBS} communication protocol. However, for outgoing connections we have not found such problem, due to the management of the different vendor's native drivers provided by \ac{JDBC}, which is introduced at the end of this section. 

\subsection{MySQL Database System}
MySQL is nowadays the most popular Open Source SQL \ac{RDBMS} \cite{mysqlmanual}. Data is stored following a relational storage model, where data is represented as tuples, and grouped into relations. The main storage structure managed in this type of database are tables, which can be linked together by establishing relationships governed by rules, e.g. one-to-one, one-to-many, many-to-many, etc. 

The MySQL server is one of the main components in the \ac{DBMS}. It is a client/server system which consists of a multi-threaded SQL server which supports different backends, several different client programs and libraries, administrative tools, and a wide range of application programming interfaces (APIs) \cite{mysqlmanual}. Its main functionality we discuss in this diploma thesis is the protocol it supports for I/O operations between the client and the server. The MySQL communication protocol has changed over time and over the \ac{DBMS} version upgrades, leading to different new user authentication methods, new data types, etc. In this diploma thesis we cover the MySQL versions 5.x support. Due to the compatibility of the native \ac{JDBC} drivers along the different versions, the supported protocol in our prototype is full compatible with the last released \ac{JDBC} MySQL native driver.

The MySQL communication protocol is used between the MySQL client and server. Implementations of the protocol can be found in the MySQL server, the MySQL native driver Connector/J (Java implemented) and in the MySQL proxy. As it is described in Figure \ref{fig:mysqlprotocol}, the whole communication process between a MySQL client and a MySQL server is divided into three phases: connection phase, authentication phase, and command phase, and their main transferred information unit are MySQL packets. The MySQL packet configuration and the supported data types are described in Chapter \ref{chap:implementation}.

\begin{figure}[htb]
	\centering
		\includegraphics[clip, scale=0.8]{./gfx/mysqlprotocol.pdf}
	\caption[MySQL Communication Protocol]{MySQL communication protocol in the four communication phases \cite{mysqlmanual}}
	\label{fig:mysqlprotocol}
\end{figure}

During the connection phase, the client connects via \ac{TCP} to the port where the main MySQL server thread listens on (commonly used port 3306). In the connection and authentication phases, the MySQL server sends to the client an initial handshake packet, containing information about the server, server flags, and a password challenge. The client responds with his access credentials and communication configuration flags. When the authentication succeeds, the command phase is initiated. This phase is actually where the operations on the database or on the server take place, e.g. server configuration, querying, statements execution, etc. The connection between the client and the server must be always ended in the client side, except for internal errors in the server where the communication is interrupt and lost. 

In this diploma thesis we extend a Java implementation of the MySQL protocol, which is described in more detail in Chapter \ref{chap:relatedworks}, and adapt it for its integration and communication in ServiceMix-mt.

\subsection{PostgreSQL Database System}

PostgreSQL is known as an \ac{ORDBMS}. An \ac{ORDBMS} is quite similar to the a \ac{RDBMS} model explained in the last section, but its main difference is that it also supports the object-oriented database model, where objects are stored in database schemas can be accessed using the \ac{SQL}. 

The PostgreSQL \ac{DBMS} also implements a client/server model for its I/O operations in the database. In contrast to the MySQL server, the PostgreSQL server defines the following cycles depending on the state of the connection: start-up, query, function call, copy, and termination \cite{postgresqlmanual}. During the start-up phase, the client opens a connection and directly provides its user name and the database he wants to connect. This information identifies the particular protocol version to be used. The server responds with an authentication challenge which the client must fulfill. 

The MySQL's SQL command phase is in this server denoted as a query cycle. A query cycle is initiated with the reception of an SQL command, and terminated with the response of the query execution. 

The function call cycle allows the client with execute permissions to request a direct call of an existing function in the system's catalog. The copy cycle switches the connection into a distinct sub-protocol, in order to provide a high-speed data transfer between the client and the server. 

The termination of a successful or failed client/server communication is handled in the termination cycle, which involves the transfer of a termination packet from the client to the server in the successful case, and from the server to the client when the termination is due to a failure. 
 
\subsection{MySQL Proxy}

The MySQL Proxy is an application which supports the MySQL communication protocol between one or more MySQL clients and MySQL servers \cite{mysqlproxy}. In a distributed storage system where different clients connect to different servers a proxy which acts as a communication intermediary may significantly increase the overall performance. The MySQL proxy supports communication management between users, communication monitoring, load balancing, transparent query alteration, etc. Oracle releases a MySQL proxy which supports MySQL 5.x or later, and implemented in the C programming language. 

Integrating a MySQL server into a \ac{ESB} collisions with the main concept of an \ac{ESB} as an intermediary technology between services. For this reason, in this diploma thesis we integrate and extend a Java version of a MySQL proxy developed by Continuent Inc.: Tungsten Connector \cite{tungstenwiki}.

\subsection{Java Database Connectivity}

\ac{JDBC} is widely used in the connection to databases in the Java programing language. JDBC technology allows programers to use the Java programming language to exploit "Write Once, Run Anywhere" capabilities for applications that require access to enterprise data.\cite{jdbcspec}. Its management of different vendor-specific native drivers allows businesses not to be locked in any proprietary architecture, but to be able to connect to different databases simply by specifying the driver's name and the connection properties in the \ac{JDBC} URL. 

The \ac{JDBC} Driver Manager or DataSource Object implements the selection of the appropriate vendor's native driver specified in the \ac{JDBC} URL. However, the vendor's native driver must be installed prior to execution. 

In this diploma thesis we take advantage of this technology in order to enable our final prototype to support a multi-protocol database outgoing communication (from the prototype to external \ac{DBS}).  


\FloatBarrier

\section{NoSQL Databases}
\label{sec:fundamentalsnosqldb}  

\ac{RDBMS}s ensure data persistency over time and provide a wide set of features. However, the functionalities supported require a complexity, which is sometimes not needed for some applications, and harms important requirements in Web applications or in \ac{SOA} based applications, e.g. throughput. \ac{NoSQL} data stores aim to improve the efficiency of large amount of data storage while reducing its management cost \cite{nosqlcomputerworld}. NoSQL databases are designed to support horizontal scalability without relying on the highly available hardware \cite{strauchnosql}. In a Cloud storage environment where the user sees the available computing and storage resources as unlimited, a \ac{NoSQL} support in a Cloud storage environment might be adequate.

\ac{NoSQL} \ac{DBS} operate as a schema-less storage system, allowing the user to access, modify or freely insert his data without having to make first changes in the data structure \cite{nosql2012}. Cloud providers provide the users with an \ac{API} for accessing, modifying, and inserting data into his isolated container. For example, a user's Amazon Dynamo DB table and item can be accessed by its RESTful \ac{API}, or by installing at the user's side application the Amazon Web Services SDK \cite{amazondynamodb}. Furthermore, it provides the users through its Web-based management console the available management operations. 

Due to the growth of the \ac{NoSQL} support along different Cloud vendors, in this diploma thesis we provide a multi-tenant and transparent communication support for \ac{NoSQL} backend data stores in different Cloud providers. In the following sections we introduce the categorization of the different \ac{NoSQL} databases we aim to support in this diploma thesis, mentioning and giving examples of Cloud data stores available nowadays in the market.

\subsection{Key-value Databases}

In a key-value datastore elements are uniquely identified by an id, which the data store does not take into account its type, and are simply stored as a \ac{BLOB} . A user can get the value for the key, put a value for the key, or delete a key from the data store \cite{nosql2012}. Its storage model can be compared to a map/dictionary \cite{strauchnosql}. Products offering this data storage model in a Cloud infrastructure are Amazon DynamoDB \cite{amazondynamodb}, Google Cloud Storage \cite{googlecloudstorage}, Amazon SimpleDB  \cite{amazonsimpledb} , Amazon S3 \cite{amazons3}, etc. In this diploma thesis we mainly focus on the following key-value data stores: DynamoDB, and Google Cloud Storage.

Amazon DynamoDB's data model includes the following concepts: tables, items, and attributes \cite{amazondynamodb}. The attributes are a key-value, where the value is binary data. Attributes are stored in items, and these are stored in tables. Items stored in a table can be retrieved by referencing its unique id. The number of attributes is not limited by Amazon, but each item must have a maximum size of 64 KB. Accessing stored data in this data store can be mainly done in two ways: using the functionalities provided by the AWS SDK, or using the Cloud storage RESTful \ac{API}. 

Google Cloud Storage's data model includes the following concepts: buckets and objects \cite{googlecloudstorage}. Buckets contain on or more objects. The objects are identified within a bucket with its unique id. Users can perform I/O operations on both buckets and objects. For this purpose, Google Cloud storage provides RESTful \ac{API}.

In this diploma thesis we use an \ac{ESB} for accessing transparently tenant's databases migrated to the Cloud. Servicemix-mt provides multi-tenant \ac{HTTP} support \cite{gomez2012}. Therefore, we reuse and extend the multi-tenant \ac{HTTP} \ac{BC} in order to provide dynamic routing between the different data stores.

\subsection{Document Databases}

Document databases can be considered as a next step in improving the key-value storage model. In this storage model, documents are stored in the value part of the key-value store, making the value content examinable \cite{nosql2012}. Documents with different schemas are supported in the same collection, and can be referenced by the collection's key or by the document's attributes. One of the main difference in the attributes specification regarding \ac{RDBMS} is that in document stores document's attributes cannot be null. When there is an attribute without value, the attribute does not exist in the document's schema. Products implementing this data storage model are Apache CouchDB, MongoDB, etc. \cite{couchdb} \cite{mongodb}.

Mongo DB defines two storage structures: collections and documents \cite{mongodb}. A specific database contains one or more collections identified by its unique id. A specific collection stores one or more documents. Collections and documents stored in a database can be accessed, inserted and modified using the RESTful \ac{API} supported by the database system.

Apache CouchDB defines two storage structures: databases and documents. Data stored in CouchDB are \ac{JSON} documents. The main difference between this two described databases is that MongoDB implements a two step access to the documents: database, collection, and document. Apache CouchDB provides a RESTful \ac{API} for I/O operations.

This databases are not offered by Cloud providers like Amazon or Google, but as a software which can be deployed in user instances, e.g. Amazon EC2 AMI \cite{amazonec2}. 

\subsection{Column-family Stores}

One of the most known Column-family data stores is Cassandra. Column-family data stores store data in column families (groups of related columns which are often accessed together) as rows that have many columns associated with a row key \cite{nosql2012}. This approach allows to store and process data by column instead of by row, providing a higher performance when accessing large amount of data, e.g. allowing the application to access common accessed information in less time.

Cassandra has as its smallest unit of storage the column, which consists of a timestamp and a name-value pair where the name acts as a key \cite{nosql2012}. As in the relational model, a set of columns form up a row, which is identified by a key. A column family is a collection of similar rows. The main difference with the relational model is that each of the rows must not have the same columns, allowing the designer and the application consuming large amounts of data to customize the columns in each row, and the rows in each column family.

Cassandra is not shipped with a RESTful API for I/O operations. However, there are several open-source services layers for Cassandra, e.g. Virgil \cite{virgil}.

\FloatBarrier

\section{JBIMulti2}
\label{sec:jbimulti2}  

A multi-tenant management system must fulfill several requirements, such as data and performance isolation between tenants and users, authentication, specification of different user roles, resources usage monitoring, etc. In a \ac{JBI} environment, endpoint and routing configurations files are packed in \ac{SU}s, and the latter in \ac{SA}s for deployment. However, there is a lack of user-specific data during deployment. Muhler solves this problem in JBIMulti2 by injecting tenant context in the \ac{SA} packages, making them tenant-aware \cite{Muhler2012}. 

\begin{figure}[htb]
	\centering
		\includegraphics[clip, scale=0.5]{./gfx/systemoverview_jbimulti2.pdf}
	\caption[JBIMulti2 System Overview]{JBIMulti2 System Overview \cite{Muhler2012}} 
	\label{fig:jbimulti2}
\end{figure}

The architecture of the JBIMulti2 system is represented in Figure \ref{fig:jbimulti2}. We can distinguish two main parts in the system: business logic and resources. JBIMulti2 uses three registries for storing configuration and management data. When a tenant (or a tenant user) is registered, an unique identification number is given to them and stored in the Tenant Registry. Both Tenant Registry and Service Registry are designed for storing data of more than one deployed application. The former for storing tenant information and the latter for providing a dynamic service discovery functionality between the different applications accessed through the \ac{ESB}. The Configuration Registry is the key of the tenant isolation requirement of the system. Each of the stored tables are indexed by the tenant id  and user id value. In this thesis we need tenant information during runtime. We reuse and extend the databases schemas produced by Muhler, specifically the Service Registry.

The system provides a user interface for accessing the application's business logic. Through the business logic, the management of tenants can be done by the system administrator or the management of tenant's users can be done by the tenants. Furthermore, when deploying the different tenant's endpoint configurations packed in \ac{SA}s, the system first makes modifications in the zip file for adding tenant context information and then communicates with the Apache ServiceMix instance by using a \ac{JMS} Topic to which all the ServiceMix instances are subscribed to. The \ac{JMS} management service in ServiceMix deploys the received \ac{SA} injected in the received \ac{JMS} message using the administration functionalities provided in ServiceMix. The communication between the business layer and the ServiceMix instance is unidirectional. When successful deployment, the endpoint is reachable by the tenant. When an error occur during deployment, an unprocessed management message is posted in a dead letter queue.

JBIMulti2 requires the previous installation of components, e.g. JOnAS server, PostgreSQL, etc. The initialization of the application is described in both Chapter \ref{chap:validationevaluation} and in the JBIMulti2 setup document \cite{JBIMulti2Man}.
\section{Cloud Data Migration Application}
\label{sec:clouddatamigrationtool}  

The Cloud Data Migration Application provides support to the user before and during the data migration process to the Cloud. It contains a registry of different Cloud data hosting solutions and its properties, which are used during the decision process. The decision process consists in selecting the Cloud provider which best fits the user's operational and economical interests, and in detecting incompatibilities with the selected target data store. The different steps of the migration process are shown in Figure \ref{fig:cloudmigrateapp}. 

\begin{figure}[htb]
	\centering
		\includegraphics[clip, scale=0.4]{./gfx/clouddatamigrationtool.png}
	\caption[Cloud Data Migration Application - Cloud Data Migration Process]{Cloud Data Migration Process \cite{bachmann2012}} 
	\label{fig:cloudmigrateapp}
\end{figure}

In the \term{data layer pattern selection} and \term{adapt data access layer steps}, the user must specify how to connect to the data store his data is migrated to, and provide the necessary information to establish the connection. The extension of ServiceMix-mt for enabling Cloud data access support allows the user to select this prototype for transparently access the migrated data. 

\FloatBarrier
\section{Apache JMeter}
\label{sec:jmeter}  

Apache JMeter is a Java-based application which provides support for load testing and performance measurement \cite{jmeter2013}. It provides support for different communication protocols, e.g. \ac{HTTP}, \ac{SOAP}, database via \ac{JDBC}, etc. A multi-protocol support enables the application to be used for testing different layers of an application, e.g. presentation layer, and database layer. Furthermore, the user can configure in JMeter different load parameters, e.g. number of threads, iterations, etc., in order to build a heavy load simulation to run on the backend server. Simulation results are presented in structured formats for posterior analysis. 

In this diploma thesis we provide an evaluation of the behavior of the final prototype. Due to the \ac{JDBC} functionality supported by JMeter, we use it to generate the load test cases which are run on ServiceMix-mt.

\FloatBarrier

\input{relatedworks/relatedworks}
\section{Specification}
\label{sec:evaluationspecification}



\subsection{Evaluation Requirements}
\label{sec:evalrequirements}

% different message size, reduce loose couple being able to specify different types of messages
% invoke the endpoints with concurrent users and invoke endpoints concurrently
% be able to set the number of concurrent users and concurrent endpoints to invoke at the same time
% include multi-tenancy awareness in the benchmark for the different scenarios
% structured data as the output for analysis, for throughput and response time for the different messages requests for the different endpoints
% has to be done on top of the androitbenchmark, so that we can reuse and extend their benchmark
% system measurements of cpu and memory in structured data. also the measurement of the heap size (real memory consumtion) in the process
% monitoring of number of outgoing requests and incoming requests into the backend service, wireshark

In this student thesis we provide a performance analysis on the integration of the multi-tenant aware approaches in ServiceMix and measure the impact on the performance of the extended prototype. For this purpose, we need to fix which measurements we use for the evaluation. The driver should perform the following measurements: response time (measured in milliseconds) and throughput (measured in number of messages sent per second) respect to a backend service, and CPU and memory usage of the system hosting the instance of ServiceMix. The evaluation has to be done in different scenarios, each of them sending different messages number and sizes, for different multi-tenant and non multi-tenant aware endpoint configurations, as described in Table \ref{tab:evaluation} \cite{EvalESB}.

\begin{table}[htbp]
\centering
\begin{tabular}{llll}

	\toprule
	 Number of Endpoints 		& Messages Size	& ServiceMix Instances		& Multi-tenancy awareness 		\\
	 \midrule
	 
	 1 						& 0.5 / 1 KB 		& 1						& mt and non-mt									\\
	  						& 				& 2						& non-mt											\\
	 2 						& 0.5 / 1 KB 		& 1						& mt and non-mt									\\
	  						& 				& 2						& non-mt											\\
	 4 						& 0.5 / 1 KB 		& 1						& mt and non-mt									\\
	  						& 				& 2						& non-mt											\\
	 10 						& 0.5 / 1 KB 		& 1						& mt and non-mt									\\
	  						& 				& 2						& non-mt											\\														
	 
	\bottomrule
\end{tabular}
\caption[ServiceMix evaluation performance scenarios]{Specification of the different scenarios to be evaluated. In both multi-tenant and non multi-tenant aware evaluations, one user per endpoint / tenant is configured. \\ \term{Legend: mt (multi-tenant aware), non-mt (non multi-tenant aware)}}
	\label{tab:evaluation}
\end{table}

AndroitLogic has developed a performance analysis driver which fulfills most of the above requirements in different scenarios \cite{androit2012}. In our evaluation, we extend the Direct Proxy scenario from the AndroitLogic  \ac{ESB} Performance benchmark \cite{androit2012}. However, it doesn't achieve one of the main requirements of this student thesis: multi-tenant aware messaging and concurrent invocation between endpoints. Those two requirements should be included in an extended version of the primitive driver. Furthermore, the extension should be utilized with different \ac{ESB} solutions and must be user-friendly configurable for the different scenarios. The output of the driver measurements should be analyzed, therefore the output data must be in structured format. 

\subsection{Evaluation Overview}
\label{sec:evaluationoverview}

% main picture of the overview of the system for evaluating the esb and explain it with detais of the hardware setup of the vm
% describe a little bit the scenarios that were fixed
% 
In the Section \ref{sec:requirements} we have described the requirements that the evaluation should fulfill and the needed modifications in the utilized benchmark. As exposed in Figure \ref{fig:evaluationoverview}, the evaluation is conformed by three main independent systems. We must ensure, for analyzable purposes, that we approximate as much as possible to a Web service standard real scenario: service requester invokes a backend service and both request and response are routed through the network. In our evaluation we must utilize the \ac{ESB} as a mediator between the service requester and provider. In the first system (VM0), both service requestor and provider are deployed. The communication measurements are taken in two different components: throughput and response time in the AndroitLogic driver, while the number of incoming and outgoing requests, as well as the visualization of the messages, have to be monitored in an independent monitoring component. 

In the second and third systems (VM1 and VM2 respectively) one instance of ServiceMix is deployed for routing the messages between the AndroitLogic driver and the backend service. A monitor component must perform the counting of the incomming and outgoing requests to and from the \ac{ESB}, and a system monitor component should measure the \ac{ESB}'s resources consumption. The connection between the components in VM0 and VM2 is represented with a dashed line, because the VM2 is only use for non multi-tenant aware scenarios. Similarly, we have connected the components in VM0 with the components in VM1 with a continuos line, because this connection is used in both multi-tenant and non multi-tenant aware scenarios. 

The JBIMulti2 application is used for deployment of the \ac{SA}s which pack the endpoint configurations in the \ac{SU}s. However, we do not include the JBIMulti2 application in our overview because we do not evaluate the JBIMulti2 performance, but the multi-tenant and non multi-tenant ServiceMix independently from the JBIMulti2 application.

\begin{figure}[htb]
	\centering
		\includegraphics[width=0.7\textwidth, trim=0.0cm 0.0cm 0.0cm 0.0cm, clip]{./gfx/evaluationoverview.pdf}
	\caption[Performance Evaluation Components Overview]{Overview of the components used for the \ac{ESB} performance evaluation. \textbf{Note:} In the evaluation two different monitors are used. For communication the monitoring requires the counting and visualization of the incoming and outgoing requests. For system monitoring, the CPU and Memory usage should be measured.}
	\label{fig:evaluationoverview}
\end{figure}

\FloatBarrier

\FloatBarrier
\section{ESB Performance Evaluation Architecture}
\label{sec:esbevaluationdesign}

% explain the figure
% the evaluation is done for SOAP over HTTP protocols based on a in only message exchange patter, so that we are only masuring how much time does it take to reach the backend service. explain that the dashed lines are because those calls are optional, this means, we divide in to more than one scenario, each scenario testing 1, 2, 4, and 10 endpoints
% for the two instances of servicemix, we provide 10 endpoints in total, and 5in each instance, and the load is divided into the endpoitns. we start with the scenario with 2 endpoints
% explain a little bit the java bench androitlogic driver
% explain that we will use differente messages for the scenarios so that we can decouple them from the scenarios, being able to modify messages independently. results are in a format (I can paste an example of a result from the java benchmark) and then the driver provided by androit to be able to convert it to a csv file
% for concurrent calls between endpoints, we are going to create as many instances of the java benchmark as many endpoints we want to call concurrently by using shell scripting and the & (background task). Explain how many messages we are sending, in factor 2..., and the warmup phase of the esb, so that we ensure that the heap mamory is cleaned (garbage collector)
% system monitoring for the %CPU and %sys memory using the top command and then converting the output data to csv
% jconsole usage for heap measurements
% wireshark just for counting the packages that arrived and pack lost
% to fill the messages, we fill it with random characters and create a <attachment> part in the body

AndroitLogic has developed in their \ac{ESB} Performance Evaluation Round 3 a load generator for different scenarios. After analyzing its main features, we found it suitable for our work, but only if we can include tenant-awareness in the execution. We evaluate the \ac{SOAP} over \ac{HTTP} communication protocol in both native ServiceMix \ac{HTTP} \ac{BC} and in the multi-tenant \ac{HTTP} \ac{BC}. With this we want to evaluate not only the performance of the \ac{ESB} solution we are using in our Cloud infrastructure, but also the penalty caused by the multi-tenant awareness implementation. The \ac{SOAP} over \ac{HTTP} protocol is well known for its usage in Web services. In this evaluation we use as a backend Web service an Echo Service which logs the received requests. For this purpose, we must push the scenarios as close as possible to a real Web service consumption. Therefore, we divide the evaluation system in two virtual machines connected by a network (see Figure \ref{fig:evaluationarchitecture}). 


%%%%%%%%%%%%%%%%%%%%%%%%%%%%%
\begin{figure}[htb]
	\centering
		\includegraphics[width=.95\textwidth, trim=0.0cm 0.0cm 0.0cm 0.0cm, clip]{./gfx/evaluationarchitecture.pdf}
	\caption[ESB Performance Evaluation Architecture]{Architectural overview of the components used for the evaluation of the \ac{ESB} performance. \textbf{Note:} We evaluate only ServiceMix, not the integrated version of ServiceMix with the JBIMulti2 application, in order to be able to perform a direct comparison between the multi-tenant and the non multi-tenant ServiceMix.}
	\label{fig:evaluationarchitecture}
\end{figure}
%%%%%%%%%%%%%%%%%%%%%%%%%%%%%


The virtual machine one hosts the front and backends components: performance benchmark and the Web service. The Web service is deployed in an Apache Tomcat server. The extended performance benchmark is built of the following components: AndroitLogic driver, shell scripts and data converters. The AndroitLogic driver support concurrent users invoking the same endpoint, but not concurrent users between two or more endpoints. Furthermore, it does not support message modification for including tenant information. For this purpose, we have designed the shell scripts which can give support on those two requirements (see Figure \ref{fig:evaluationarchitecture}). In the first place, the shell script modifies or does not modify the message which will be sent by the driver. In the second place, we perform concurrent invocations between endpoints by creating several Unix background tasks of the driver. Each of the tasks results can be dumped in a shared file between the driver instances. However, the results come in non structured format for analysis. Therefore, we convert the data using a converter provided by AndroitLogic \cite{androit2012}. For monitoring the packet lost rate, we will listen on the server's port where the Web service listens with a well known monitoring tool, Wireshark \cite{wireshark}.

We use the virtual machines two and three for hosting the ServiceMix instances. The two instances are used only in non multi-tenant scenarios. For both multi-tenant and non multi-tenant scenarios we must increase the number of concurrent calls to the endpoints. In the requirement we specify scenarios of one, two, four, and ten endpoints. The system performance measurement can be done by system commands. We provide a component which take CPU and Memory measurements and converts its output to structured data for analysis. However, the system memory usage measurements do not give variable percentages over time. The percentage shown is the one associated with the memory consumption of the JVM the \ac{ESB} runs on, which is previously reserved and fixed over time. To get more representative data, we measure the heap consumption of ServiceMix in the JVM using Java Console, which give us a better representation of the variability between the different scenarios (See Figure \ref{fig:evaluationarchitecture}). For monitoring the communication, an instance of Wireshark can also be used, but in our evaluation it is optional.


\FloatBarrier
\clearpage
\chapter{Implementation}
\label{chap:implementation}

In this chapter we describe the challenges and problems during the implementation phase to fulfill the requirements specified in Chapter \ref{chap:spec} and the design presented in Chapter \ref{chap:design} of the system. Furthermore, we discuss the incompatibilities found with components we must extend. We divide, as in the previous chapters, the implementation phase into the \ac{SQL} and \ac{NoSQL} databases support, and provide a separate section for the extensions made to JBIMulti2 and the Cache. 


\section{Structured Query Language Databases}
\label{sec:fundamentalssql}  

The \ac{SQL} stands nowadays as the standard computer database language in \ac{SQL} \ac{DBS}. \ac{SQL} is a tool for organizing, managing, and retrieving data stored by a computer database \cite{sql1999}. The \ac{SQL} language is nowadays one of well known languages in the IT sector. Thus, we introduce in the following sections in the \ac{SQL} \ac{DBS} and their specific communication protocol we use in this thesis, rather than on the language they support. 

The final prototype of this diploma thesis aims to provide support for most of the \ac{DBS} communication protocols available in the market. However, we crashed into vendor-specific communication protocol implementations along the available \ac{DBMS} in the market, rather than a common standardized communication protocol. For this reason, we provide support for incoming connections which comply the MySQL \ac{DBS} communication protocol and give the hints for supporting the PostgreSQL \ac{DBS} communication protocol. However, for outgoing connections we have not found such problem, due to the management of the different vendor's native drivers provided by \ac{JDBC}, which is introduced at the end of this section. 

\subsection{MySQL Database System}
MySQL is nowadays the most popular Open Source SQL \ac{RDBMS} \cite{mysqlmanual}. Data is stored following a relational storage model, where data is represented as tuples, and grouped into relations. The main storage structure managed in this type of database are tables, which can be linked together by establishing relationships governed by rules, e.g. one-to-one, one-to-many, many-to-many, etc. 

The MySQL server is one of the main components in the \ac{DBMS}. It is a client/server system which consists of a multi-threaded SQL server which supports different backends, several different client programs and libraries, administrative tools, and a wide range of application programming interfaces (APIs) \cite{mysqlmanual}. Its main functionality we discuss in this diploma thesis is the protocol it supports for I/O operations between the client and the server. The MySQL communication protocol has changed over time and over the \ac{DBMS} version upgrades, leading to different new user authentication methods, new data types, etc. In this diploma thesis we cover the MySQL versions 5.x support. Due to the compatibility of the native \ac{JDBC} drivers along the different versions, the supported protocol in our prototype is full compatible with the last released \ac{JDBC} MySQL native driver.

The MySQL communication protocol is used between the MySQL client and server. Implementations of the protocol can be found in the MySQL server, the MySQL native driver Connector/J (Java implemented) and in the MySQL proxy. As it is described in Figure \ref{fig:mysqlprotocol}, the whole communication process between a MySQL client and a MySQL server is divided into three phases: connection phase, authentication phase, and command phase, and their main transferred information unit are MySQL packets. The MySQL packet configuration and the supported data types are described in Chapter \ref{chap:implementation}.

\begin{figure}[htb]
	\centering
		\includegraphics[clip, scale=0.8]{./gfx/mysqlprotocol.pdf}
	\caption[MySQL Communication Protocol]{MySQL communication protocol in the four communication phases \cite{mysqlmanual}}
	\label{fig:mysqlprotocol}
\end{figure}

During the connection phase, the client connects via \ac{TCP} to the port where the main MySQL server thread listens on (commonly used port 3306). In the connection and authentication phases, the MySQL server sends to the client an initial handshake packet, containing information about the server, server flags, and a password challenge. The client responds with his access credentials and communication configuration flags. When the authentication succeeds, the command phase is initiated. This phase is actually where the operations on the database or on the server take place, e.g. server configuration, querying, statements execution, etc. The connection between the client and the server must be always ended in the client side, except for internal errors in the server where the communication is interrupt and lost. 

In this diploma thesis we extend a Java implementation of the MySQL protocol, which is described in more detail in Chapter \ref{chap:relatedworks}, and adapt it for its integration and communication in ServiceMix-mt.

\subsection{PostgreSQL Database System}

PostgreSQL is known as an \ac{ORDBMS}. An \ac{ORDBMS} is quite similar to the a \ac{RDBMS} model explained in the last section, but its main difference is that it also supports the object-oriented database model, where objects are stored in database schemas can be accessed using the \ac{SQL}. 

The PostgreSQL \ac{DBMS} also implements a client/server model for its I/O operations in the database. In contrast to the MySQL server, the PostgreSQL server defines the following cycles depending on the state of the connection: start-up, query, function call, copy, and termination \cite{postgresqlmanual}. During the start-up phase, the client opens a connection and directly provides its user name and the database he wants to connect. This information identifies the particular protocol version to be used. The server responds with an authentication challenge which the client must fulfill. 

The MySQL's SQL command phase is in this server denoted as a query cycle. A query cycle is initiated with the reception of an SQL command, and terminated with the response of the query execution. 

The function call cycle allows the client with execute permissions to request a direct call of an existing function in the system's catalog. The copy cycle switches the connection into a distinct sub-protocol, in order to provide a high-speed data transfer between the client and the server. 

The termination of a successful or failed client/server communication is handled in the termination cycle, which involves the transfer of a termination packet from the client to the server in the successful case, and from the server to the client when the termination is due to a failure. 
 
\subsection{MySQL Proxy}

The MySQL Proxy is an application which supports the MySQL communication protocol between one or more MySQL clients and MySQL servers \cite{mysqlproxy}. In a distributed storage system where different clients connect to different servers a proxy which acts as a communication intermediary may significantly increase the overall performance. The MySQL proxy supports communication management between users, communication monitoring, load balancing, transparent query alteration, etc. Oracle releases a MySQL proxy which supports MySQL 5.x or later, and implemented in the C programming language. 

Integrating a MySQL server into a \ac{ESB} collisions with the main concept of an \ac{ESB} as an intermediary technology between services. For this reason, in this diploma thesis we integrate and extend a Java version of a MySQL proxy developed by Continuent Inc.: Tungsten Connector \cite{tungstenwiki}.

\subsection{Java Database Connectivity}

\ac{JDBC} is widely used in the connection to databases in the Java programing language. JDBC technology allows programers to use the Java programming language to exploit "Write Once, Run Anywhere" capabilities for applications that require access to enterprise data.\cite{jdbcspec}. Its management of different vendor-specific native drivers allows businesses not to be locked in any proprietary architecture, but to be able to connect to different databases simply by specifying the driver's name and the connection properties in the \ac{JDBC} URL. 

The \ac{JDBC} Driver Manager or DataSource Object implements the selection of the appropriate vendor's native driver specified in the \ac{JDBC} URL. However, the vendor's native driver must be installed prior to execution. 

In this diploma thesis we take advantage of this technology in order to enable our final prototype to support a multi-protocol database outgoing communication (from the prototype to external \ac{DBS}).  


\FloatBarrier

\section{NoSQL Databases}
\label{sec:fundamentalsnosqldb}  

\ac{RDBMS}s ensure data persistency over time and provide a wide set of features. However, the functionalities supported require a complexity, which is sometimes not needed for some applications, and harms important requirements in Web applications or in \ac{SOA} based applications, e.g. throughput. \ac{NoSQL} data stores aim to improve the efficiency of large amount of data storage while reducing its management cost \cite{nosqlcomputerworld}. NoSQL databases are designed to support horizontal scalability without relying on the highly available hardware \cite{strauchnosql}. In a Cloud storage environment where the user sees the available computing and storage resources as unlimited, a \ac{NoSQL} support in a Cloud storage environment might be adequate.

\ac{NoSQL} \ac{DBS} operate as a schema-less storage system, allowing the user to access, modify or freely insert his data without having to make first changes in the data structure \cite{nosql2012}. Cloud providers provide the users with an \ac{API} for accessing, modifying, and inserting data into his isolated container. For example, a user's Amazon Dynamo DB table and item can be accessed by its RESTful \ac{API}, or by installing at the user's side application the Amazon Web Services SDK \cite{amazondynamodb}. Furthermore, it provides the users through its Web-based management console the available management operations. 

Due to the growth of the \ac{NoSQL} support along different Cloud vendors, in this diploma thesis we provide a multi-tenant and transparent communication support for \ac{NoSQL} backend data stores in different Cloud providers. In the following sections we introduce the categorization of the different \ac{NoSQL} databases we aim to support in this diploma thesis, mentioning and giving examples of Cloud data stores available nowadays in the market.

\subsection{Key-value Databases}

In a key-value datastore elements are uniquely identified by an id, which the data store does not take into account its type, and are simply stored as a \ac{BLOB} . A user can get the value for the key, put a value for the key, or delete a key from the data store \cite{nosql2012}. Its storage model can be compared to a map/dictionary \cite{strauchnosql}. Products offering this data storage model in a Cloud infrastructure are Amazon DynamoDB \cite{amazondynamodb}, Google Cloud Storage \cite{googlecloudstorage}, Amazon SimpleDB  \cite{amazonsimpledb} , Amazon S3 \cite{amazons3}, etc. In this diploma thesis we mainly focus on the following key-value data stores: DynamoDB, and Google Cloud Storage.

Amazon DynamoDB's data model includes the following concepts: tables, items, and attributes \cite{amazondynamodb}. The attributes are a key-value, where the value is binary data. Attributes are stored in items, and these are stored in tables. Items stored in a table can be retrieved by referencing its unique id. The number of attributes is not limited by Amazon, but each item must have a maximum size of 64 KB. Accessing stored data in this data store can be mainly done in two ways: using the functionalities provided by the AWS SDK, or using the Cloud storage RESTful \ac{API}. 

Google Cloud Storage's data model includes the following concepts: buckets and objects \cite{googlecloudstorage}. Buckets contain on or more objects. The objects are identified within a bucket with its unique id. Users can perform I/O operations on both buckets and objects. For this purpose, Google Cloud storage provides RESTful \ac{API}.

In this diploma thesis we use an \ac{ESB} for accessing transparently tenant's databases migrated to the Cloud. Servicemix-mt provides multi-tenant \ac{HTTP} support \cite{gomez2012}. Therefore, we reuse and extend the multi-tenant \ac{HTTP} \ac{BC} in order to provide dynamic routing between the different data stores.

\subsection{Document Databases}

Document databases can be considered as a next step in improving the key-value storage model. In this storage model, documents are stored in the value part of the key-value store, making the value content examinable \cite{nosql2012}. Documents with different schemas are supported in the same collection, and can be referenced by the collection's key or by the document's attributes. One of the main difference in the attributes specification regarding \ac{RDBMS} is that in document stores document's attributes cannot be null. When there is an attribute without value, the attribute does not exist in the document's schema. Products implementing this data storage model are Apache CouchDB, MongoDB, etc. \cite{couchdb} \cite{mongodb}.

Mongo DB defines two storage structures: collections and documents \cite{mongodb}. A specific database contains one or more collections identified by its unique id. A specific collection stores one or more documents. Collections and documents stored in a database can be accessed, inserted and modified using the RESTful \ac{API} supported by the database system.

Apache CouchDB defines two storage structures: databases and documents. Data stored in CouchDB are \ac{JSON} documents. The main difference between this two described databases is that MongoDB implements a two step access to the documents: database, collection, and document. Apache CouchDB provides a RESTful \ac{API} for I/O operations.

This databases are not offered by Cloud providers like Amazon or Google, but as a software which can be deployed in user instances, e.g. Amazon EC2 AMI \cite{amazonec2}. 

\subsection{Column-family Stores}

One of the most known Column-family data stores is Cassandra. Column-family data stores store data in column families (groups of related columns which are often accessed together) as rows that have many columns associated with a row key \cite{nosql2012}. This approach allows to store and process data by column instead of by row, providing a higher performance when accessing large amount of data, e.g. allowing the application to access common accessed information in less time.

Cassandra has as its smallest unit of storage the column, which consists of a timestamp and a name-value pair where the name acts as a key \cite{nosql2012}. As in the relational model, a set of columns form up a row, which is identified by a key. A column family is a collection of similar rows. The main difference with the relational model is that each of the rows must not have the same columns, allowing the designer and the application consuming large amounts of data to customize the columns in each row, and the rows in each column family.

Cassandra is not shipped with a RESTful API for I/O operations. However, there are several open-source services layers for Cassandra, e.g. Virgil \cite{virgil}.

\FloatBarrier

\section{Extensions}
\label{sec:extensions}

Apart from the components which we implement in this diploma thesis, we extend existing components developed in the works from Muhler \cite{Muhler2012}, and Uralov \cite{Uralov2012}. We extend the JBIMulti2 application developed by Muhler, and adapt the \ac{JBI} ServiceMix Registry performed by Uralov. 


\subsection{JBIMulti2} 

The multi-tenant aware administration and management application JBIMulti2 offers a list of operations through its Web service interface in order to allow tenants to deploy multi-tenant aware endpoints in ServiceMix-mt. However, it does not provide support for persisting the tenant's Cloud data stores communication meta-data. In Chapter \ref{chap:design} we describe the necessary modifications on the service registry, and the need of extending the application's Web service interface to avoid a connection from the \term{Cloud Data Migration Application} to the database system which stores the tenant configuration data. 

The Service Registry database schema is created using the Java Persistence \ac{API}, and annotating the Java classes with meta-data. We extend the Java class \term{ServiceAssembly}, which stores the \ac{SA}s deployed by the tenant. The \term{ServiceAssembly} class contains the schema representation, and the data storage and retrieval operations.

The Web service interface developed by Muhler \cite{Muhler2012}, and extended by Uralov \cite{Uralov2012}, contains several management operations, e.g. deploy service assembly, create tenant, create user, deploy \ac{JBI} BC, etc., which allows the system administrator and tenants to perform management operations on ServiceMix-mt. The \term{Cloud Data Migration Application} described in Chapter \ref{chap:fundamentals} retrieves from the user the backend database system meta-data, such as access credentials, database name, etc. The application may be hosted on a separate server as JBIMulti2. In order to avoid a direct connection from the \term{Cloud Data Migration Application} to the Service Registry, which contains sensible data, we extend the JBIMulti2 Web service interface, and provide the operations for registering the tenant's backend Cloud data store meta-data. 

\subsection{Cache}

The cashing support in ServiceMix-mt is provided in the \ac{OSGi} component \ac{JBI} ServiceMix Registry. Uralov provides a set of operations to set up a cache instance, store elements, and retrieve stored elements \cite{Uralov2012}. However, the \ac{JBI} ServiceMix Registry \ac{OSGi} bundle libraries are not registered as an \ac{OSGi} service. Therefore, this component is not accessible from third party \ac{OSGi} bundles in the \ac{OSGi} container. We modify its original \term{BundleActivator} class and include the \ac{OSGi} service registration to allow its usage from external \ac{OSGi} bundles.

\lstinputlisting[label={lst:ehcacheconfig},caption={[EhCache Configuration for SQL Support]Eh cache configuration for \ac{SQL} support.},style=xml]{./gfx/ehcacheconfig.xml}

The cache instances are created on the Ehcache 2.6.0 component \cite{ehcache}, which provides support to store serializable objects indexed by a key. A cache instance configuration in the Ehcache component must be specified in an \ac{XML} file, which is described in Listing \ref{lst:ehcacheconfig}. However, the cashing support is not multi-tenant aware. We implement a dynamic creation of multi-tenant aware cache keys in order to ensure isolation between the cashed tenant's information, and data (see Listing \ref{lst:cachekey}). 

\FloatBarrier

\input{validationEvaluation/validationevaluation}
%\section{Evaluation}
\label{sec:evaluation}

Extensions implemented in ServiceMix-mt may affect on its performance, and system's resources consumption. Therefore, in this section we focus on evaluating, in the first place, the operative system's resources the \ac{ESB} consumes. In the second place, we evaluate the difference between connecting directly to the backend Cloud or local data store (without transparency to the user), and the utilization of a transparent connection through the Cloud-Enabled Data Access Bus. For evaluation purposes we utilize the TPC-H benchmark for generating the data, and Apache JMeter for performing the load tests. 

\subsection{TPC-H Benchmark}

The TPC-H benchmark is a set of libraries written in the C language which provide support for generating large volumes of data for populating the database system which wants to be evaluated \cite{tcpbenchmark}. Furthermore, it generates queries with a high degree of complexity to be executed on the databases where the generated data is stored. In this diploma thesis we utilize the TPC-H benchmark to generate the data which is stored in the local MySQL database system, and in the backend MySQL database instances hosted in Amazon RDS. The data generated by the benchmark varies in size, and is stored in a database which follow the schema described in Figure \ref{fig:tpchschema}. 

\begin{figure}[htb]
	\centering
		\includegraphics[clip, scale=0.5]{./gfx/tpchdbschema.pdf}
	\caption[TPC-H Database Schema]{TPC-H database schema generated \cite{tcpbenchmark}.}
	\label{fig:tpchschema}
\end{figure}

The TPC-H data and queries generator operations are mainly managed in two executables, which are generated by building the library with the \term{make} command: \term{dbgen}, and \term{qgen}. The former has as input option the size, which we set in this diploma thesis to 1GB, while the latter generates a set of queries for the generated data. After the data and queries generation, the user must manually import the data into the database system to evaluate. For more information about the processes of generating data and queries we attach in the prototype a short tutorial which is adjusted to the operative system we use in the FlexiScale's VM: Ubuntu 10.04.

\FloatBarrier

%4 cpus, 8 GB ram, ubuntu 10.04.4_64bits, ip 109.231.70.234, java version "1.6.0_24"
%servicemix-mt 4.3.0 with memory min 256 and max 1GB
%mysql server 5.1.67 in local machine
%mysql server 5.5 in amazon rds
%apache jmeter 2.9
% jconsole
% tpch benchmark version 2.15.0, and we use the data generator, which generates aprox 1GB of data
% amazon rds
\subsection{Evaluation Overview}

The system we set up for the evaluation is built up of three subsystems which are hosted in different infrastructures. The first subsystem resides in a private network provided in the University of Stuttgart, and identified by the hostname \term{dyn139.iaas.uni-stuttgart.de}. A local machine with Apache JMeter 2.9 installed is connected to the network. In the evaluation we aim to approach as much as possible to the motivating scenario: database layer migrated to the Cloud, and accessing the remote database system from the on-premise application's data access layer. Therefore, we decide to perform the queries in a load generator program from a private network (see Figure \ref{fig:evaluationoverview}). 

We classify in Figure \ref{fig:evaluationoverview} as the second subsystem the Amazon RDS database systems which are host in the Amazon Cloud \cite{amazonrds}. Amazon provides the user with the option to deploy his database instances in different regions. We select the N. Virginia region, create the tenant 2 database on a db.m2.xlarge database instance, and transfer the table schemas and data generated by the TPC-H benchmark. For billing purposes we utilize the \term{EDUStudentGrantsSpring-Summer2012} \cite{awseducational}  

\begin{figure}[htb]
	\centering
		\includegraphics[clip, scale=0.6]{./gfx/evaluationoverview.pdf}
	\caption[Evaluation Architecture Overview]{Evaluation architecture overview for one tenant, two users, and local and remote \ac{SQL} Cloud data store.}
	\label{fig:evaluationoverview}
\end{figure}

The third subsystem is hosted in a VM image in the FlexiScale Cloud infrastructure \cite{flexiscale}. In an Ubuntu 10.04 64 bits operative system with a Java 6 VM we install the following components (see Figure \ref{fig:evaluationoverview}): 
\begin{itemize}
	\item JOnAS 5.2.2.
	\item PostgreSQL 9.1.1.
	\item Extended version of JBIMulti2
	\item A MySQL 5.1 database system which hosts the tenant 1 database.
	\item Extended version of ServiceMix-mt: we modify its minimum and maximum heap consumption allowance, and set it to minimum 256 MB, and maximum 1 GB.
	\item TPC-H data and query generator.
	\item Resource measurements component
\end{itemize}

The resource measurements component measures the CPU utilization of the ServiceMix-mt Java process. Its memory consumption is measured using the JConsole program provided by the JVM (Java Virtual Machine). In this evaluation we are interested in measuring the heap consumption, rather than the memory which is consumed by the JVM. The TPC-H is not used for measurement purposes, but only for data and query generation purposes.  

The evaluation scenarios are defined in Table \ref{tab:evaluation}. We define 8 different scenarios, which follow the following criteria:
\begin{itemize}
	\item Direct connection to the backend database system vs. connection through ServiceMix-mt.
	\item Number of Users, number of concurrent requests per user, and number of requests per user. 
	\item Data stored in a local MySQL database (in the same instance as ServiceMix-mt) vs. data stored in an Amazon RDS MySQL database instance.
\end{itemize}

\lstinputlisting[label={lst:querytpc},caption={[Evaluation Query]Query included in the MySQL requests of the load generator.},style=xml]{./gfx/tpcquery.txt}

The evaluation includes the measurements for the following performance units: throughput (requests per second), data transfer speed (KB/sec), CPU utilization (\%), and Memory utilization (MB). Furthermore, the scenarios are run utilizing the same data retrieval query, which is detailed in Listing \ref{lst:querytpc}.

\begin{table}[htbp]
\centering
\begin{tabular}{llllll}

	\toprule
	Id 		& User num.	& Threads / user		& Req. / thread 		& Backend database	& Through ESB\\
	 \midrule
	 
	 1.1.1.1 						& 2 		& 20						& 100			& Local MySQL			& Yes\\
	 1.1.1.2 						& 2 		& 20						& 100			& Local MySQL			& No\\
	 1.1.2.1 						& 2 		& 50						& 200			& Local MySQL			& Yes\\
	 1.1.2.2 						& 2 		& 50						& 200			& Local MySQL			& No\\
	 2.1.1.1						& 2		& 20						& 100			& MySQL Amazon RDS	& Yes\\
	 2.1.1.2						& 2 		& 20						& 100			& MySQL Amazon RDS	& No	\\
	 2.1.2.1						& 2 		& 50						& 200			& MySQL Amazon RDS	& Yes\\
	 2.1.2.2						& 2 		& 50						& 200			& MySQL Amazon RDS	& No\\
	 
	 
	\bottomrule
\end{tabular}
\caption[CDASMix Evaluation Performance Scenarios]{Specification of the different scenarios to be evaluated. \textbf{Note: } Evaluated the performance for connection through CDASMix and direct to Amazon RDS \cite{amazonrds}.}
	\label{tab:evaluation}
\end{table}

\FloatBarrier

\subsection{Evaluation Analysis}

In this subsection we discuss and present the evaluation results obtained from the execution of the scenarios described in Table \ref{tab:evaluation}. Before getting into the discussion, we point out a problem in the evaluation, which leads us to discard the scenario 2.1.2.2 (see Table \ref{tab:evaluation}). The throughput obtained in the scenario 2.1.1.2 (see Table \ref{tab:evaluation}) is in average 3,3 requests per minute, and lasts approximately 10 hours. This fact makes us delete the last scenario execution, due to the low performance obtained, which we assume that it is related with the Quality of Service assigned to the student grants credits profile in Amazon RDS. 

As it can be seen in Figure \ref{fig:throughput}, utilizing ServiceMix-mt as the database layer component for communicating with a MySQL database system locally deployed lowers the throughput in a 32,74 \% for 2 users, 20 concurrent threads per user, and 100 requests per thread. However, when the thread and request number are increased, the throughput exponentially decreases, as we reach the limit of concurrent connections supported in the MySQL CDASMix Proxy. When we compare it with a direct connection to a locally deployed MySQL database system, we can see that the concurrent requests are better handled by the MySQL server. In case of avoiding cashing support in ServiceMix-mt, the throughput would considerably decrease when increasing the load. When the distance between the different layers of the application increases, e.g. accessing a database layer deployed in a Cloud infrastructure A, and the database system is located in a Cloud infrastructure B, the number of requests per second decreases (see scenario 2.1.1.1 in Figure \ref{fig:throughput}). However, the difference we obtain from hosting the database layer on premise, but accessing a database system off-premise is 98,64 \% worse, if we compare if with the access through ServiceMix-mt. We must denote that in these scenarios there is a high temporal proximity of equal requests. Therefore, the cashing mechanism in the system increases considerably the throughput when accessing data hosted off-premise through ServiceMix-mt.
 
\begin{figure}[htb]
	\centering
		\includegraphics[clip, scale=0.8]{./gfx/throughput.png}
	\caption[Evaluation Analysis - Throughput]{Throughput (requests per second) for the different scenarios described in Table \ref{tab:evaluation}.}
	\label{fig:throughput}
\end{figure}

\FloatBarrier

The amount of KB per second transmitted in the different scenarios correlates with the tendency which can be seen in throughput (see Figures \ref{fig:throughput} and \ref{fig:kbs}).

\begin{figure}[htb]
	\centering
		\includegraphics[clip, scale=0.8]{./gfx/kbs.png}
	\caption[Evaluation Analysis - Transmission Speed]{Transmission speed (KB per second) for the different scenarios described in Table \ref{tab:evaluation}.}
	\label{fig:kbs}
\end{figure}

\FloatBarrier

Memory utilization maintains stable along the different scenarios. We observe that in non of the scenarios the maximum heap size (1 GB) is reached in maximum or average values (see Figure \ref{fig:memory}). We obtain a lower memory utilization for the scenarios where data retrieval from a backend Cloud data store is involved. This difference relies on the network latency of having the data hosted in a database system which is not in the same network (in our evaluation, the database instance hosted in Amazon RDS), and the low network latency of having the database system in the same machine as the database layer. A greater number of threads handling the routing requests are blocked due to a higher response time when accessing a remote database system. 

\begin{figure}[htb]
	\centering
		\includegraphics[clip, scale=0.8]{./gfx/memory.png}
	\caption[Evaluation Analysis - Memory Utilization]{Memory utilization (MB) for the different scenarios described in Table \ref{tab:evaluation} where ServiceMix-mt is involved.}
	\label{fig:memory}
\end{figure}

The same difference seen in the memory utilization can be observed in the CPU consumption (see Figure \ref{fig:cpuutilization}). When the requests are executed on a local database system, the response time per request is highly lower than a response time from a remote database. Therefore, the CPU utilization averages and maximum values are closer to each other. However, when accessing the MySQL database instance in Amazon RDS, the maximum values correlate with the scenarios 1.1.1.1 and 1.1.2.1, but in average the CPU utilization is lower due to a higher number of blocked threads.

\begin{figure}[htb]
	\centering
		\includegraphics[clip, scale=0.8]{./gfx/cpuutilization.png}
	\caption[Evaluation Analysis - CPU Utilization]{CPU utilization (\%) for the different scenarios described in Table \ref{tab:evaluation} where ServiceMix-mt is involved.}
	\label{fig:cpuutilization}
\end{figure}

\FloatBarrier
\chapter{Outcome and Future Work}
\label{chap:outcome}
%What have we done??
% we have evaluated the different technologies that were used and the different components which were separately implemented. multi-protocol multi-tenant communication in servicemix (soap http, jms, email and camel), included authentication in the esb, confidentiality and integrity is still missing. integrated version with the taxi scenario and separate endpoints configuration for both testing scenarios that we run, one for the taxi scenario and one for testing the endpoints individually. two echo services for testing the implemented approaches and used afterwords for the esb performance analysis. an analysis of the performance of the esb solution we are using and we have compared the penalty in the performance because of our modifications. we have also compared the actual performance improvement of distributing the load between two instances of servicemix and realized that is not a 50 % improved, this leads to a higher cost and less proportional improvement.  

Migration of one or more application layers to the Cloud aims to reduce the cost in the required IT infrastructure to host and maintain such layer within an organization. Adaptations on both migrated and non migrated layers are a must when part of an application is migrated to a Cloud infrastructure. In this diploma thesis we start from the basis of a partial migration of the application, particularly the database layer. The database layer includes the operations which provide data storage and retrieval support. Furthermore, the software, hardware and maintenance required to host and maintain this layer require an economical budget substantially greater than the needed for the business, or presentation layers of an application. Migrating the application's data to the Cloud requires rewiring the database connections to the backend Cloud data store, and adaptating the upper layers to match the operations, and data representations supported. Providing transparent communication support for accessing and storing data on application's databases in the Cloud demands a multi-protocol, and multi-tenant component. In this diploma thesis we extend a multi-tenant aware \ac{ESB} in order to utilize it as the database layer of the application, and access multiple Cloud data stores providing \ac{SQL} and \ac{NoSQL} databases. 

In Chapter \ref{chap:fundamentals} we present the necessary background about the technologies we use, and the components we reuse in this diploma thesis, e.g. JBIMulti2 \cite{Muhler2012}, \ac{JBI} and \ac{OSGi} frameworks, etc. Furthermore, we categorize the databases systems which are supported in the prototype, and subcategorize them based on their storage model and communication protocols. 

After researching on the \ac{SQL} and \ac{NoSQL} database systems properties in Chapters \ref{chap:fundamentals} and \ref{chap:relatedworks}, we find that most of database communication protocols are not standardized, and differ along the different database vendors. Therefore, we are forced in Chapter \ref{chap:design} to develop components which adjust to specific communication protocols: \ac{HTTP}, and MySQL. The research described in Chapter \ref{chap:relatedworks} leads us to find a lack of standardization in the communication at the TCP level of the \ac{SQL} database vendors, but the existence of components which support other communication protocols for incoming requests, e.g. \ac{HTTP}, and utilize via \ac{JDBC} the different database vendors' native driver to forward them to the backend database system \cite{jboss2011}. However, approaches in this direction forces the developer to adjust their data access layer, whose adaptations we aim to minimize when utilizing our system. \ac{NoSQL} Cloud data store providers show a lack of standardized naming, and provide the database users with their drivers for I/O operations. In order to address the lack of a standardized naming and access in the \ac{NoSQL} providers, we categorize in the system's registry the user's databases meta-data into different categories and subcategories, and access the backend data stores via the \ac{HTTP} protocol supported in ServiceMix-http-mt \cite{gomez2012}. 

The functional and non-functional requirements the system must fulfill are described in Chapter \ref{chap:spec}. After analyzing the requirements, providing an overview of the system, and specifying the necessary use cases, we move to the design of the prototype in Chapter \ref{chap:design}. We divide the design into the design of common components, and database specific components for the following databases types: \ac{SQL}, and \ac{NoSQL} databases. Apache ServiceMix-mt is used as the main component in the system for enabling a multi-protocol and multi-tenant communication support between backend databases. We design two different \ac{NMF}s' content for requests for enabling a dynamic routing of requests between the backend database systems. We extend the JBIMulti2 registries schemas to support the storage of tenant's migrated database configuration data. Components which implement the different databases systems communication protocols, e.g. MySQL and \ac{HTTP}, and components enabling routing between the multi-tenant aware consumer and provider endpoints are presented in Chapter \ref{chap:design}. However, the \ac{SQL} database support is limited in the system to one specific database system: MySQL. Separate components can be implemented to provide support for more \ac{SQL} database vendors, e.g. PostgreSQL or Oracle. Furthermore, the \ac{NoSQL} databases support is limited to the backend databases which support the \ac{HTTP} communication protocol. 

The implementation, validation, and evaluation of the system which complies with the requirements and the design of Chapters \ref{chap:spec} and \ref{chap:design} respectively, is explained in Chapters \ref{chap:implementation} and \ref{chap:validationevaluation}. We first validate the system by creating backend databases which contain custom data, and data generated by the TPC-H benchmark \cite{tcpbenchmark}. After configuring the communication configuration in CDASMix through the JBIMulti2 Web service interface, the tenant can communicate with the backend Cloud data store. Communication configuration should be done in the future through a user friendly Web interface, by integrating the \term{Cloud Data Migration Tool} and JBIMulti2 Web interfaces. 

For evaluating the advantages and disadvantages of utilizing CDASMix as the communication component in a database layer, we create an evaluation baseline for a backend MySQL database, run the different evaluation scenarios, and discuss its results. Applications with a high number of I/O operations often suffer a high performance decrease \cite{cashing2012}. Therefore, we enhance our prototype with a temporal multi-tenant cashing mechanism. Results describing the behavior of the system with a high load of data requests is described in Chapter \ref{chap:validationevaluation}, and demonstrate the advantages of cashing when accessing databases in the Cloud through CDASMix. An evaluation of the system's communication performance between \ac{NoSQL} databases is recommended in future works.

Further future works involve a secure authentication mechanism in CDASMix, as well as horizontal scalability of the system, and query transformation. The former involves implementing in the CDASMix MySQL Proxy the authentication mechanisms supported in the MySQL database server, and including password verification in the authentication phase in the multi-tenant \ac{HTTP} \ac{BC}. Horizontal scalability can be obtained between multiple instances of ServiceMix-mt building the system. Future versions of CDASMix can provide separate but connected ServiceMix-mt instances for routing requests for \ac{SQL} and \ac{NoSQL} database systems, or implementing a load balancer between the multiple instances building the data access layer. The developed version of CDASMix does not provide query and data transformation between different database versions, or database vendors. However, the system's design and implementation is extensible. A transformation component can be inserted between the endpoints in ServiceMix-mt.



%future work
%web gui, for both jbimulti2 and cloud data migration tool.-
% connection of the cloud data migration tool with the jbimulti2-
% benchmarking of the nosql
% benchmarking of the sql for more providers, and trying different configurations of database location and products and providers
% authentication in the system with the password
% support for incoming postgresql
% multiple instances of an esb connected to each other, and separate for example, one esb for sql, and one esb for nosql
% transformation of queries, and data

%The utilization of an \ac{ESB} as the main piece of middleware for \ac{SOA} in a Cloud environment forces multi-tenancy awareness to be a must in its requirements. This student thesis integrates the two main approaches for enabling multi-tenancy in an open source \ac{ESB}: multi-tenant aware messaging and multi-tenant aware administration and management, as well as analyzes and compares the performance of the native and extended \ac{ESB} solution in different scenarios, and produces as its main outcome an integrated version of the taxi application \cite{4CaaSt}. 

%In Chapter \ref{chap:fundamentals} we first provide the needed background on the technologies, communication protocols, and the main components this student thesis work with: ServiceMix and JBIMulti2 \cite{ASM}, \cite{Muhler2012}. After acquiring the main knowledge of the solutions, we investigate in Chapter \ref{chap:relatedworks} different solutions which support multi-tenancy, and analyze approaches which have been already taken into account. Furthermore, we discuss the supported functionalities of the AndoitLogic load generator driver, and the possibility of its reuse in our performance analysis. The identification of requirements and the system overview presented in Chapter \ref{chap:spec} guide us to perform the design of the different components of this student thesis in Chapter \ref{chap:design}. The design leads to a multi-tenant and multi-protocol aware version of ServiceMix, supporting three communication protocols: \ac{SOAP} over \ac{HTTP}, \ac{JMS}, and E-mail. Furthermore, we provide an integration design for the taxi application v2.0 prototype and we reengineer most of the implemented communication approaches in order to improve the system's performance and to add new functionalities, e.g. tenant context data structure modification, tenant authentication, and tenant-aware isolated endpoints in the \ac{ESB}. One of the main requirements in a Cloud infrastructure is security. We implement tenant authentication but not tenant data integrity and confidentiality. The tenant context information sent to and from the \ac{ESB} must be encrypted in future versions of the \ac{ESB}.

%Chapter \ref{chap:implementation} describes the challenges and approaches we faced for both the integration with the taxi scenario and the extension of the different \ac{JBI} \ac{BC}s. As discussed in Chapter \ref{chap:implementation}, one of the main goals we have in the improved version of the taxi application is to maximize the \ac{ESB} usage between components and to integrate a multi-tenant aware \ac{ESB} with non multi-tenant aware components which build part of the taxi application, e.g. \ac{BPEL} processes under Orchestra, CMF and GoogleDirections components, etc. This contrast forced us to perform changes in some of the taxi application components in order to adapt multi-tenancy at the communication level. Future versions of the taxi application should support multi-tenancy awareness in its components, and we consider that a bidirectional connection between JBIMulti2 and the taxi companies Web interfaces should be set in order to retrieve the tenant context information. Furthermore, the only communication protocol which actually supports the taxi application is the \ac{SOAP} over \ac{HTTP}. We have extended both \ac{JMS} and Mail \ac{JBI} \ac{BC}s for supporting a multi-protocol communication between customers and taxi drivers in a future version of the taxi application. This aspect directed us to build two different testing environments, one for the taxi application and one for the individual testing of the extended \ac{BC}s described in Chapter \ref{chap:test}.

%For analytical purposes after implementation, we perform an evaluation of the performance in Chapter \ref{chap:performanceevaluation} of both native and extended versions of ServiceMix. This student thesis reuses and extends an existing \ac{SOAP} over \ac{HTTP} \ac{ESB} performance benchmark. We adapt the benchmark to support multi-tenancy and evaluate the obtained results from different scenarios. However, we could not perform this analysis on more that one communication protocol. In the future it would be interesting to run the same scenarios on the \ac{XML} over \ac{JMS} communication protocol. Those results can give the \ac{ESB} administrator a better output for offering the communication protocols which best execute in our ServiceMix version. Furthermore, we perform the evaluation of one important scenario in a Cloud infrastructure: dividing the load between more than one ServiceMix instance by emulating a load balancer. The results showed that the performance is not significantly increased with the increase of the number of endpoints, and this approach can be not worth its expenditure. However, as we discussed, we emulate load balancing. For more than one instance of \ac{ESB} a load balancer should be integrated to the extended system. 

%Finally, we have integrated a multi-tenant \ac{ESB} which connect different endpoints via different protocols, e.g. external consumers with external providers. However, data is nowadays the most important asset of any business \cite{CHONGA2006}. The offering of a data-as-a-service solution in a Cloud environment where data can accessed through \ac{SOA} mechanisms, ables the use of the \ac{ESB} as a data access layer. With this approach data can be accessed from everywhere just by communicating with the \ac{ESB} and without worrying about the underlying architecture, e.g. database vendor, connection drivers. 



% Things to put: security in communication, JMS performance testing, using the ESB as Data Access Layer, optimization in the system resources usage of the actual implementation, load balancing between two or more instances of the esb, integration of the taxi scenario to communicate with jms or email, jms management of dominik improve it to have a bidirectional pipeline between jbimulti2 and servicemix, as well as connecting jbi multi2 with the taxi application and transmitter with security mechanisms in order to make a direct deployment of bcs from there or to just send the tenant and user information


\begin{appendix}


\chapter{Components}

%This chapter lists \ac{XSD}s that WS-Policy Assertion language and Rules XML file must conform. Furthermore, there are given several policy documents of Cloud data stores that are used for validation.

\section{CDASMix MySQL Proxy}
\label{appendix:cdasmixmysqlproxy}

The MySQL proxy \ac{OSGi} bundle is implemented on the Continuent Tungsten Connector \cite{tungstenwiki}, which is a Java MySQL proxy which directly connects with the backend MySQL database system. We extend and adapt this proxy in order to integrate it with ServiceMix, aggregate transparency, multi-tenant awareness, cashing, and dynamic connection with the backend Cloud data sources.

\begin{figure}[htb]
	\centering
		\includegraphics[clip, scale=0.4]{./gfx/mysql-osgi/mysql-proxy-v3.pdf}
	\caption[ServiceMix-mt MySQL OSGi Bundle]{OSGi bundle providing MySQL support in ServiceMix-mt}
	\label{fig:mysqlclassdiagram}
\end{figure}

\FloatBarrier

\vspace*{0.5cm}

%\lstinputlisting[label={lst:policy_language_syntax},caption={[Syntax of WS-Policy Assertion Language Schema]Syntax of WS-Policy Assertion Language Schema.},style=xml]{./gfx/master_thesis/cdhs-ws.xml}

%\vspace*{1cm}

%\vspace*{0.5cm}

%\lstinputlisting[label={lst:policy_language_schema},caption={[CDHS WS-Policy Assertion Language Schema]CDHS WS-Policy Assertion Language Schema.},style=xml]{./gfx/master_thesis/cdhs_properties.xsd}

%\vspace*{1cm}

\section{CDASMix Camel JDBC}
\label{subsec:cdasmixcameljdbc}

The \term{cdasmixjdbc} component is a custom component which is built and deployed as an \ac{OSGi} bundle in ServiceMix-mt. It provides support for connections with backend \ac{SQL} Cloud data stores, and message marshaling and demarshaling.

\begin{sidewaysfigure}[htb]
	\centering
		\includegraphics[clip, scale=0.6]{./gfx/cdasmix-camel-jdbc/cdasmix-camel-jdbc.pdf}
	\caption[ServiceMix-mt Camel CDASMix-JDBC Component]{OSGi bundle and Camel component providing JDBC support in ServiceMix-mt}
	\label{fig:cdasmixjdbcclassdiagram}
\end{sidewaysfigure}

\vspace*{0.5cm}

\FloatBarrier

%\lstinputlisting[label={lst:rules_schema},caption={[Post-Processing Rules XML Schema]Post-Processing Rules XML Schema.},style=xml]{./gfx/master_thesis/rules.xsd}

%\vspace*{1cm}

%\section{Multi-tenant ServiceMix HTTP Binding Component}
%\label{appendix:httpmtbc}

%to be filled
%\vspace*{0.5cm}

%\lstinputlisting[label={lst:googleCloudSQL-policy},caption={[Google Cloud SQL Service Provider Policy]Google Cloud SQL Service Provider Policy.},style=xml]{./gfx/master_thesis/provider_policies/GoogleCloudSQL.wspolicy}

%\vspace*{1cm}

%\vspace*{0.5cm}

%\lstinputlisting[label={lst:sqlDatabase-policy},caption={[SQL Database Service Provider Policy]SQL Database Service Provider Policy.},style=xml]{./gfx/master_thesis/provider_policies/SQLDatabase.wspolicy}

%\vspace*{1cm}
\FloatBarrier


\clearpage
\chapter{Messages}

%This chapter lists \ac{XSD}s that WS-Policy Assertion language and Rules XML file must conform. Furthermore, there are given several policy documents of Cloud data stores that are used for validation.
In this chapter we provide an overview of the requests which are sent to, and received from the extended ServiceMix-mt. For MySQL requests we provide the \ac{TCP} packets which are transferred between the application, ServiceMix-mt, and the backend MySQL database system. For NoSQL requests we present messages samples which are in \ac{JSON} format, but its content varies among the different backend Cloud data store providers. 

\section{Normalized Message Format Content Description}
\label{appendix:nmfcontent}

In this section we provide an overview of the data structures which are sent in the \ac{NMF}. The sections of the \ac{NMF} where the data and meta-data are sent are the \term{properties}, and the \term{attachment}.  In the Listing \ref{lst:nmfcontent} we detail the contents sent in each of the sections, and the data structures in which the data and meta-data are stored.

%\vspace*{0.5cm}

%\lstinputlisting[label={lst:policy_language_syntax},caption={[Syntax of WS-Policy Assertion Language Schema]Syntax of WS-Policy Assertion Language Schema.},style=xml]{./gfx/master_thesis/cdhs-ws.xml}

%\vspace*{1cm}
\vspace*{0.5cm}
%%%%%%%%%%%%%%%%%%%%%%%%%%%%%
\lstinputlisting[label={lst:nmfcontent},caption={[Data and Meta-data Detail in the Normalized Message Format]Detail of the content and data structures used to send the requests' data and meta-data.},style=ebnf]{./gfx/NMF_v3_0.txt}
%%%%%%%%%%%%%%%%%%%%%%%%%%%%%
\vspace*{1cm}

\section{MySQL TCP Stream}
\label{appendix:messagemysql}

In this section we provide two \ac{TCP} streams which are captured with the \term{ngrep} program for UNIX \cite{ngrep}. The first stream captures the \ac{TCP} packets on port 3311, where the MySQL component in ServiceMix-mt listens for incoming connections (see Listing \ref{lst:tcpstream3311}). The second stream captures the \ac{TCP} packets on port 3306, where the a locally deployed MySQL server listens for incoming connections (see Listing \ref{lst:tcpstream3306}).

%%%%%%%%%%%%%%%%%%%%%%%%%%%%%
\lstinputlisting[float=htb,label={lst:tcpstream3311},caption={[TCP Stream for a MySQL Communication Captured on Port 3311]TCP Stream for a MySQL communication captured on port 3311 with the program \term{ngrep} \cite{ngrep}.},style=ebnf]{./gfx/tcpstreamoutput3311.txt}
%%%%%%%%%%%%%%%%%%%%%%%%%%%%%

%%%%%%%%%%%%%%%%%%%%%%%%%%%%%
\lstinputlisting[float=htb,label={lst:tcpstream3306},caption={[TCP Stream for a MySQL Communication Captured on Port 3306]TCP Stream for a MySQL communication captured on port 3306 with the program \term{ngrep} \cite{ngrep}.},style=ebnf]{./gfx/tcpstream3306.txt}
%%%%%%%%%%%%%%%%%%%%%%%%%%%%%

\vspace*{0.5cm}

%\lstinputlisting[label={lst:policy_language_syntax},caption={[Syntax of WS-Policy Assertion Language Schema]Syntax of WS-Policy Assertion Language Schema.},style=xml]{./gfx/master_thesis/cdhs-ws.xml}

\vspace*{1cm}

\vspace*{0.5cm}

%\lstinputlisting[label={lst:policy_language_schema},caption={[CDHS WS-Policy Assertion Language Schema]CDHS WS-Policy Assertion Language Schema.},style=xml]{./gfx/master_thesis/cdhs_properties.xsd}

\vspace*{1cm}

\FloatBarrier


%\section{JSON in POST Request}
%\label{subsec:jsonpost}

%to be filled

\vspace*{0.5cm}

%\lstinputlisting[label={lst:rules_schema},caption={[Post-Processing Rules XML Schema]Post-Processing Rules XML Schema.},style=xml]{./gfx/master_thesis/rules.xsd}

\vspace*{1cm}

\FloatBarrier

\clearpage

\end{appendix}

\bibliographystyle{alphancex}
\bibliography{literatur/literatur}

All links were last followed on March 21, 2013.

\clearpage
\pagestyle{empty}
\vspace{9cm}
\begin{center}
\begin{minipage}{11cm}
\vspace{6cm}

\textbf{\Large Acknowledgement}\\\\
I am heartily thankful to my supervisor Steve Strauch from the University of Stuttgart for his encouragement, guidance and support in all the phases of this diploma thesis. I am also grateful to Dr. Vasilios Andrikopoulos for his advices and useful tips.
Special thanks to my family, friends and girlfriend for their moral support.

\vspace{1cm}
Santiago G\'omez S\'aez

\end{minipage}
\end{center}

\clearpage
\pagestyle{empty}
\vspace{9cm}
\begin{center}
\begin{minipage}{11cm}
\vspace{6cm}

\textbf{\Large Declaration}\\\\
\vspace{0.4cm}

All the work contained within this thesis,
except where otherwise acknowledged, was
solely the effort of the author. At no
stage was any collaboration entered into
with any other party.
\vspace{1cm}

Stuttgart, 22nd March 2013 \hspace{1cm}--------------------------------\\
\phantom{Stuttgart, March 22 2013} \hspace{1.3cm} (Santiago G\'omez S\'aez)
\end{minipage}
\end{center}

\end{document}

\section{Informationsrückgewinnung}
\label{sec:Informationsrückgewinnung}

In the last decades our world has become more and more interconnected. This interconnection added to the increase of the available bandwidth and the change in business models have forced IT Systems to fulfill its demands, leading to its reorganization into a public utility which offers public services, like water, electricity, etc. The \ac{NIST} defines Cloud computing as "a model for enabling ubiquitous, convenient, on-demand network access to a shared pool of configurable computing resources (e.g., networks, servers, storage, applications, and services) that can be rapidly provisioned and released with minimal management effort or service provider interaction"  \cite{NIST2011}. The Cloud computing model is composed of five characteristics:
	\begin{enumerate}
		\item On-demand self-service: a Cloud user consumes the Cloud provider's computing capabilities automatically without the need of human interaction. 
		\item Broad network access: computing capabilities are available via the network and can be accessed using standard mechanisms.
		\item Resource pooling: computing capabilities in the Cloud provider side are virtualized to serve multiple consumers simultaneously using a multi-tenant model. The Cloud consumer generally has no sense of the provided resources.
		\item Rapid Elasticity: computing and storage resources can be dynamically (and in some cases automatically) provisioned and released to respond to the actual consumers' demand.
		\item Measured Service: resources' usage is monitored and measured in a transparent way to the Cloud consumer and provider for control and optimization purposes.
	\end{enumerate}

The control that the Cloud consumer has over the computer resources in a Cloud provider infrastructure is defined in three service models: \term{\ac{SaaS}}, \term{Platform-as-a-Service (\ac{PaaS})} and \term{\ac{IaaS}}. \term{\ac{SaaS}} provides to the Cloud consumer access and usage of Cloud provider's applications running on a Cloud infrastructure. The consumer has no control over the underlying infrastructure where the application he uses is deployed. The customer can only control individual application's configurations during his usage of it. \term{\ac{PaaS}} provides the customer with the needed capabilities to deploy applications which's programming language, required libraries, services and tools are supported by the provider. The consumer has no control over the underlying infrastructure where he deploys the application. \term{\ac{IaaS}} is the model which gives most control to the consumer. Thus, the consumer is able to deploy and run arbitrary software and has the control over operating systems, storage and deployed applications, but has no management or control on the underlaying Cloud infrastructure. 

Although the three service models described above provide both data computation and storage capabilities for the consumer, they do not provide to the customer the possibility to directly and uniquely purchase access of storage services. In this diploma thesis we concentrate in two concrete models: \term{\ac{DBaaS}} and \term{\ac{STaaS}}. Cloud storage providers target a selected number of consumers, who process their data on-premise, but do not  want to cover the expenses of a local database system, or a backup system, among others. The Cloud storage model alleviates the need in organizations to invest in database hardware and software, to deal with software upgrades, and to maintain a professional team for its support and maintenance \cite{dbaasIyer}.  \ac{DBaaS} and \ac{STaaS} can be considered quite similar, except for one of their main distinction characteristics: their access interface. The former is the most robust data solution offered as a service, as it offers a full-blown database functionality. It can be accessed via the most common database protocols, such us MySQL, Oracle, etc, or by REST interfaces supporting \ac{SQL}. Examples of this model are Amazon RDS \cite{amazonrds} and Oracle Cloud \cite{oraclecloud}. On the other hand, the latter provides REST, \ac{SOAP} over \ac{HTTP}, or Web-based interfaces in order to perform the operations over the stored data \cite{cloudstorageWU}. Examples of this model are Amazon Dynamo \cite{amazondynamodb} , Google App Engine Datastore \cite{googleappdatastore}, and Dropbox \cite{dropbox}.

\ac{NIST} defines four deployment models in Cloud computing. A private Cloud consists in a Cloud infrastructure which is provisioned exclusively for one organization and used by the members conforming the organization. It is comparable to processing facilities that are enhanced with the Cloud computing characteristics. A community Cloud is a Cloud infrastructure where its use is limited to organizations which share the same requirements. A public Cloud infrastructure can be accessed and used by the public. It is usually offered by Cloud service providers that sell Cloud services made for general public or enterprises. Some of the Cloud consumers may process and store information which requires more control over the infrastructure in which is located, or consume public Cloud computing resources during peak loads in their private Cloud infrastructure. The hybrid Cloud model combines two or more deployment models described above and the combination remains as a unique entity.  

Cloud computing and \ac{SOA} are related styles at an architectural, solution and service level, according to IBM \cite{IBM2011}. Cloud providers expose their Cloud infrastructure as services as part of a \ac{SOA} solutions and the communication between Clouds in the Hybrid Cloud model described above can be compared to a SOA communication solution between enterprises. Cloud services are services that can be accessed by the Cloud consumers through the network. Therefore, we can deduce that the SOA model can be applied in the Cloud computing approach. As the \ac{ESB} is the central piece of \ac{SOA}, the need of the \ac{ESB} in a Cloud computing infrastructure as an integration middleware for the Cloud services is essential. 
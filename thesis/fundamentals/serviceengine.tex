\section{Service Engine}
\label{sec:serviceengine}  

A \ac{SE} can provide different kinds of services, e.g. business logic, routing, and message transformation. In this diploma thesis we will mainly concentrate on one: Apache Camel \cite{Camel2011}, which is wrapped in a ServiceMix-camel \ac{JBI} \ac{SE} in ServiceMix, and in a ServiceMix-camel-mt \ac{JBI} \ac{SE} in ServiceMix-mt  for multi-tenancy awareness..

\subsection{Apache Camel}

Apache Camel is an open-source integration framework based on known \ac{EIP} which supports the creation of routes and mediation rules in either a Java based Domain Specific Language (or Fluent API), via Spring based XML Configuration files or via the Scala DSL \cite{Camel2011}. In ServiceMix, Apache Camel is shipped in a \ac{JBI} \ac{SE}. The routing or mediation rules between two or more endpoints can be specified in an Spring Configuration file or in a \ac{POJO} file whose's class extends the Apache Camel \term{RouteBuilder} class. Route configurations deployed in ServiceMix must follow the \ac{JBI} compliance: files describing the route configuration must be packed in \ac{SU}, and the latter in a \ac{SA}. Apache Camel provides Maven archetypes which generate the needed route configuration files (in \ac{XML} or \ac{POJO} formats) where the developer can program the route between the different supported endpoints \cite{MAVEN}. The configuration in a \ac{XML} file reduces the configuration complexity to a minimum effort of the developer. However, a configuration in a \ac{POJO} class increases the developing complexity but allows the developer to provide logic, filtering, dynamic routing, etc. In the \term{RouteBuilder} class a developer can access, for example, the header of a \ac{NM} and select the target endpoint dynamically depending on the implemented logic. Furthermore, the routing patterns supported by Apache Camel are the point-to-point routing and the publish/subscribe model. 

The endpoints representation in Apache Camel is based on \ac{URI}. This allows this \ac{SE}s to integrate with any messaging transport protocol supported in the \ac{ESB}, e.g. \ac{HTTP}, \ac{JMS} via ActiveMQ, E-Mail, CXF, etc. The ServiceMix-camel \ac{JBI} \ac{SE} provides integration support between camel and \ac{JBI} endpoints. Muhler extends this component and allows dynamic internal creation of tenant-aware endpoints in the ServiceMix-camel-mt \ac{JBI} \ac{SE} \cite{Muhler2012}. The main goal of this extension is to provide an integrated environment between \ac{JBI} and camel supported endpoints. However, multi-tenancy is supported at the tenant level only in the \ac{JBI} endpoints. In this diploma thesis we aim to enable multi-tenancy not only at the tenant level, but also at the user level, as discussed in Chapters \ref{chap:spec} and \ref{chap:design}.

For enabling data access support with \ac{SQL} \ac{DBMS} in ServiceMix-mt we extend a well-known camel component: Camel-jdbc. The Camel-jdbc component enables \ac{JDBC} access to \ac{SQL} databases, using the standard \ac{JDBC} API, where queries and operations are sent in the message body \cite{cameljdbc}. This component requires the developer to statically denote the data source configuration (user, password, database name, etc.) in both the endpoint \ac{URI} and route configuration file. As discussed in Chapters \ref{chap:spec} and \ref{chap:design}, this requirement is opposite to our approach, due to the dynamism we need in creating connections to the different \ac{DBaaS} providers. We extend and produce a custom camel component: Camel-jdbccdasmix (\term{cdasmix} stands for Cloud Data Access Support in ServiceMix-mt).

\FloatBarrier
